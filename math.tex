\documentclass{ctexart}
\usepackage{xcolor}
\usepackage{geometry}
\usepackage{fancyhdr}
\usepackage{etoc}
\usepackage{tasks}
\usepackage{tipa}
\usepackage{xparse}
\usepackage{multicol}
\usepackage{enumitem}
\usepackage{amsmath, amssymb, bm}
\usepackage{mathtools}
\usepackage{physics}
\usepackage{tabularx}
\usepackage{booktabs}
\usepackage{extarrows}
\usepackage{xeCJKfntef}
\usepackage{tikz}
\usepackage[most]{tcolorbox}
\usepackage{stackengine}
\usepackage[calc]{datetime2}
\DTMsetdatestyle{iso}
\usepackage[hidelinks]{hyperref}
\hypersetup{
  bookmarksopen,
  CJKbookmarks,
}
\usepackage{bookmark}
\usepackage[xspace]{ellipsis}

%% toc style
\newcommand*{\DotsAndPage}{%
  \nobreak\hskip 5pt\leaders\hbox{%
    \bfseries\normalsize\hbox to .75ex {\hss.\hss}
  }\hfill\nobreak
  \makebox{\large\bfseries\normalsize\itshape\etocpage}\par
}
\etocsetstyle{section}
  {}
  {\leftskip 0pt \rightskip .75cm \parfillskip-\rightskip
  \pagebreak[1]\smallskip}
  {\Large\rmfamily\bfseries\scshape{\etocname}\DotsAndPage}
  {}
\etocsetstyle{subsection}
  {}
  {\leftskip 1cm \rightskip .75cm \parfillskip-\rightskip
  \pagebreak[1]\smallskip}
  {\large\rmfamily\bfseries\scshape{\color{teal}\etocname}\DotsAndPage}
  {}
\renewcommand{\etoccolumnsep}{4em}
\etocmulticolstyle[3]{
  \noindent\bfseries\Large
  \leaders\hrule height1pt\hfill
  \MakeUppercase{Table of Contents}
}

\ctexset{
  section/format=\raggedright\bfseries\huge,
}

\newcolumntype{L}{>{\raggedright\arraybackslash}X}
\newcolumntype{R}{>{\raggedleft\arraybackslash}X}

\renewcommand{\ellipsisgap}{0.1em}
\setlength\parindent{0pt}

\newtcbox{\reviewtoday}{
  colframe=red!20, colback=yellow!10, fontupper=\small\itshape,
  size=small,
}

% \ExplSyntaxOn
% \NewDocumentCommand{\defineclist}{mm}
 % {
  % \clist_clear_new:c { l__#1_clist }
  % \clist_set:cn { l__#1_clist } { #2 }
 % }
% \ExplSyntaxOff

\makeatletter
\ExplSyntaxOn
\cs_new_protected:Npn \custom_label:n #1
  {
    \iow_shipout:Nx \@auxout { \token_to_str:N \newlabel{#1}{{}{\thepage}{}{#1}{}} }
    \hypertarget{#1}{}
  }

\NewDocumentCommand {\customlabel} {m}
  {
    \custom_label:n {#1}
  }
\ExplSyntaxOff
\makeatother
\endinput

\ExplSyntaxOn
\bool_new:N \g_review_bool
\newcount\datediff

% #1 clist #2 date to be compared
\prg_new_protected_conditional:Nnn \review_if:Nn {T, F, TF}
  {
    \DTMsavedate{recent}{#2}
    \DTMsavenow{today}
    \DTMsaveddatediff{today}{recent}{\datediff}
    \bool_gset_false:N \g_review_bool
    \clist_map_inline:Nn #1
      {
        \int_compare:nT {\datediff == ##1}
          {
            \bool_gset_true:N \g_review_bool
            \clist_map_break:
          }
      }
    \bool_if:nTF {\g_review_bool}
      {\prg_return_true:}
      {\prg_return_false:}
  }
\ExplSyntaxOff

\everymath{\displaystyle}
\setlength\parindent{0pt}

\geometry{
  left=2cm,
  right=2cm,
  top=2.5cm,
  bottom=2.5cm,
}

\newcommand{\mathblank}{\CJKunderline[hidden]{四个汉字}}

\newcommand{\mathhint}[1]{
  \tcbset{highlight math style={colframe=teal, colback=teal!10}}
  \tcbhighmath[boxrule=0.6pt]{#1}
}

\newcommand{\matherror}[1]{
  \tcbset{highlight math style={colframe=red, colback=red!10}}
  \tcbhighmath[boxrule=0.6pt]{#1}
}

\newtcbox{\mathidea}{
  colframe=teal, colback=teal!10, on line,
}

\newtcbox{\mathtype}{
  colframe=teal, colback=teal!30!cyan,
  size=small, on line,
}

\newtcbox{\mathlink}{
  colframe=teal, colback=teal!30!yellow,
  size=small, on line,
  fontupper=\sffamily,
}

\newtcolorbox{mathideabox}{
  colframe=teal, colback=teal!10, breakable,
}

\newtcolorbox[use counter=step]{step}[1]{
  title={Step \arabic{step}: #1},
  breakable,
}

\newcounter{step}

\ExplSyntaxOn

\AtEndDocument{
  \math_show:
}

\cs_new_protected:Nn \math_def_seq:n
  {
    \seq_new:c { g__math_saved_all_#1_seq }
    \seq_new:c { g__math_saved_review_#1_seq }
    \seq_new:c { g__math_saved_more_#1_seq }
  }

\tl_new:N       \l__math_date_tl
\tl_new:N       \l__math_type_tl
\tl_new:N       \l__math_info_tl
\tl_new:N       \l__math_content_ques_tl
\tl_new:N       \l__math_number_style_tl
\int_new:N      \l__math_mistake_number_int
\int_new:N      \l__math_count_int
\dim_new:N      \l__math_ques_after_title_dim
\bool_new:N     \l__show_all_bool
\clist_new:N    \l__math_review_point_clist
\math_def_seq:n { date }
\math_def_seq:n { type }
\math_def_seq:n { info }
\math_def_seq:n { content_ques }
\math_def_seq:n { content_solu }
\math_def_seq:n { mistake_number }

\clist_set:Nn     \l__math_review_point_clist { 2,4,7,15,30 }
\bool_set_false:N \l__show_all_bool

\keys_define:nn { mathques }
  {
    show~all     . bool_set:N = \l__show_all_bool,
    show~all     . default:n  = { true },
    number~style . tl_set:N   = \l__math_number_style_tl,
    after~title  . dim_set:N  = \l__math_ques_after_title_dim,
  }

\NewDocumentCommand { \mathsetup } { +m }
  {
    \keys_set:nn { mathques } { #1 }
  }

\prg_new_protected_conditional:Nnn \math_review_if:n { T, F, TF }
  {
    \review_if:NnTF \l__math_review_point_clist { #1 }
      { \prg_return_true: }
      { \prg_return_false: }
  }

% #1 date #2 seq name #3 data to save
\cs_new_protected:Nn \math_save:nnn
  {
    \bool_if:NT \l__show_all_bool
      {
        \seq_gput_right:cn { g__math_saved_all_#2_seq } { #3 }
      }
    \math_review_if:nT { #1 }
      {
        \seq_gput_right:cn { g__math_saved_review_#2_seq } { #3 }
      }
  }

% #1 做错次数 #2 做错日期 #3 题目类型
  \NewDocumentEnvironment { mathques } { O{} D(){1} m m +b }
  {
    \tl_clear_new:N \l__math_date_tl

    \tl_set:Nn \l__math_date_tl { #3 }

    #5

    \math_save:nnn { #3 } { date } { #3 }
    \math_save:nnn { #3 } { type } { #4 }
    \math_save:nnn { #3 } { mistake_number } { #2 }
  } { }

\NewDocumentCommand { \info } { +m }
  {
    \math_save:nnn { \l__math_date_tl } { info } { #1 }
  }

\NewDocumentEnvironment { ques } { +b }
  {
    \math_save:nnn { \l__math_date_tl } { content_ques } { #1 }
  } { }

\NewDocumentEnvironment { solu } { +b }
  {
    \math_save:nnn { \l__math_date_tl } { content_solu } { #1 }
  } { }

\cs_new_protected:Nn \math_show:
  {
      { \math_show_content_template:n { review } }
    \bool_if:NT \l__show_all_bool
      { \math_show_content_template:n { all } }
  }

\cs_new_protected:Nn \math_show_content_template:n
  {
    \math_show_content:nn { #1 } { ques }
    \clearpage
    \math_show_content:nn { #1 } { solu }
  }

% #1 all/review #2 ques/solu
\cs_new_protected:Nn \math_show_content:nn
  {
    \int_zero:N \l__math_count_int
    \seq_map_inline:cn { g__math_saved_#1_content_#2_seq }
      {
        \setcounter{step}{0}
        \int_incr:N \l__math_count_int
        \math_set:nn { #1 } { \l__math_count_int }
        \cs_if_exist_use:c { math_wrap_content_#2:nn } { #1 } { ##1 }
      }
  }

% #1 all/review/more #2 int
\cs_new_protected:Nn \math_set:nn
  {
    \tl_set:Nn \l__math_date_tl
      { \math_seq_item:nnn { #1 } { date } { #2 } }
    \tl_set:Nn \l__math_type_tl
      { \math_seq_item:nnn { #1 } { type } { #2 } }
    \tl_set:Nn \l__math_info_tl
      { \math_seq_item:nnn { #1 } { info } { #2 } }
    \tl_set:Nn \l__math_content_ques_tl
      { \math_seq_item:nnn { #1 } { content_ques } { #2 } }
    \int_set:Nn \l__math_mistake_number_int
      { \math_seq_item:nnn { #1 } { mistake_number } { #2 } }
  }

% #1 all/review/more #2 date/type... #3 int
\cs_new:Nn \math_seq_item:nnn
  {
    \seq_item:cn { g__math_saved_#1_#2_seq } { #3 }
  }

% #1 all/review... #2 content
\cs_new_protected:Nn \math_wrap_content_ques:nn
  {
    \customlabel{#1_ques:\int_use:N \l__math_count_int}
    \begin{tcolorbox}
      {
        \l__math_number_style_tl
        \int_use:N \l__math_count_int
      }.~
      \mathtype{ \l__math_type_tl }
      \mathlink{
        \hyperref[#1_solu:\int_use:N \l__math_count_int]{
          前往答案
        }
      }
      \par\vspace{\l__math_ques_after_title_dim}
      #2
    \end{tcolorbox}
  }

\cs_new_protected:Nn \math_wrap_content_solu:nn
  {
    \par
    \customlabel{#1_solu:\int_use:N \l__math_count_int}
    {
      \l__math_number_style_tl
      \int_use:N \l__math_count_int
    }.~
    \mathtype{ \l__math_type_tl }
    \mathlink{
      \hyperref[#1_ques:\int_use:N \l__math_count_int]{
        返回题目
      }
    }
    \par\vspace{\l__math_ques_after_title_dim}
    {\sffamily\bfseries 题目:}\l__math_content_ques_tl
    \par{\sffamily\bfseries 答案:}
    #2 \par
  }

\ExplSyntaxOff

\mathsetup{
  show all=false,
  number style={ \zihao{3}\color{teal}\itshape\bfseries },
  after title=1em,
}

% \mathsetup{show all}

\begin{document}

\begin{mathques}{2020-09-05}{极限}
\begin{ques}
  极限$\lim_{n \to \infty} \left( \frac{n+1}{n-2} \right) ^n=$\mathblank.
\end{ques}
\begin{solu}
  \begin{align*}
    \lim_{n \to \infty} \left( \frac{n+1}{n-2} \right) ^n
    &= \lim_{n \to \infty} \frac{\left( 1 + \frac{1}{n} \right) ^n}{\left( 1 -
    \frac{2}{n} \right) ^n}\\
    & \mathhint{\lim_{n \to \infty} \left( 1 + \frac{1}{n} \right) ^n = e}
    \\
    & = \frac{\lim_{n \to \infty} \left( 1 + \frac{1}{n} \right) ^n}{\lim_{n
    \to \infty} \left( 1 - \frac{2}{n} \right) ^n}\\
    & = \frac{e}{e^{-2}} = e^3
  \end{align*}
\end{solu}
\end{mathques}

\begin{mathques}{2020-09-05}{极限}
\begin{ques}
  求极限$\lim_{n \to \infty} (\sqrt{n + \sqrt{n}} - \sqrt{n - \sqrt{n}})$.
\end{ques}
\begin{solu}
  \begin{align*}
    \lim_{n \to \infty} (\sqrt{n + \sqrt{n}} - \sqrt{n - \sqrt{n}})
    &= \lim_{n \to \infty} \frac{(n + \sqrt{n}) - (n - \sqrt{n})}{(n +
    \sqrt{n}) + (n - \sqrt{n})}\\
    &= \lim_{n \to \infty} \frac{2}{\sqrt{1 + \frac{1}{\sqrt{n}}} + \sqrt{1 -
    \frac{1}{\sqrt{n}}}}\\
    &= \frac{2}{\matherror{\sqrt{1} + \sqrt{1}}} = 1
  \end{align*}
\end{solu}
\end{mathques}

\begin{mathques}{2020-09-05}{极限}
\begin{ques}
  设$x_{n + 1} = \sqrt{2 + x_n} (n = 1, 2, \dotsc), x_1 = \sqrt{2}$, 证明
  $\lim_{n \to \infty} x_n$ 存在, 并求$\lim_{n \to \infty} x_n$.
\end{ques}
\begin{solu}
  \mathidea{单调有界数列必有极限}
  \begin{step}{证明单调性(3种方法)}
    \begin{enumerate}
      \item 令$f(x) = \sqrt{2 + x}$ 证明$f(x)$的单调性
      \item 作差$x_{n} - x_{n - 1} = \sqrt{2 + x_{n - 1}} - x_{n - 1} = \frac{2
        + x_{n - 1} - x_{n - 1}^2}{\sqrt{2 + x_{n - 1}} + x_{n - 1}}$\par
        $2 + x_{n - 1} - x_{n - 1}^2 > 2 + x_{n - 1} - 2x_{n - 1} = 2 - x_{n
        - 1} > 0$\par
        \mathidea{方法2需要先进行Step 2证明上界为2才行}
      \item 证明$x_{n + 1} - x_n$与$x_{n} - x_{n - 1}$同号\par
        $x_{n + 1} - x_n = \sqrt{2 + x_n} - \sqrt{2 + x_{n - 1}} = \frac{x_n
        - x_{n - 1}}{\sqrt{2 + x_n} + \sqrt{2 + x_{n - 1}}}$\par
        又$x_2 - x_1 = \sqrt{2 + \sqrt{2}} - \sqrt{2} > 0$,故$x_n$单调
    \end{enumerate}
  \end{step}
  \begin{step}{证明有界性}
    \begin{mathideabox}
      \itshape 当界值难以看出来时,可以先假定有界,得出有极限然后求出根据递
      推公式求出极限得出界值再反过来想怎么证明(见Step 3)
    \end{mathideabox}
    $x_1 = \sqrt{2} < 2$, 设$x_k < 2$, 则$x_{k + 1} = \sqrt{2 + x_k} < 2$,
    故$x_n < 2$有上界
  \end{step}
  \begin{step}{求极限}
    $x_n > 0$, 设$\lim_{n \to \infty} x_n = A > 0$, 则$A = \sqrt{2 + A} \Rightarrow A = 2$
  \end{step}
\end{solu}
\end{mathques}

\begin{mathques}{2020-09-06}{极限}
\begin{ques}
  求极限$\lim_{x \to 0^+} \frac{x^x - (\sin x)^x}{x^2 \ln (1 + x)}$.
\end{ques}
\begin{solu}
\begin{align*}
  \text{原式} & = -\matherror{\lim_{x \to x^+} x^x} \lim_{x \to 0^+}
  \frac{\left( \frac{\sin x}{x} \right) ^x - 1}{x^3}\\
  &\mathhint{e^x - 1 \sim x (x \to 0)}\\
  &= -\lim_{x \to 0^+} \frac{\ln \frac{\sin x}{x}}{x^2}\\
  &= -\lim_{x \to 0^+} \frac{\ln \left( 1 + \frac{\sin x - x}{x} \right) }{x^2}
  \\
  &\mathhint{\ln (1+x) \sim x (x \to 0)}\\
  &= -\lim_{x \to 0^+} \frac{\sin x - x}{x^3}\\
  &= -\lim_{x \to 0^+} \frac{x - \frac{1}{3!} x^3 + o(x^3) - x}{x^3} = \frac{1}
  {6}
\end{align*}
\end{solu}
\end{mathques}

\begin{mathques}{2020-09-06}{极限}
\begin{ques}
  求极限$\lim_{x \to 0} \left( \frac{a_1^x + a_2^x + \dotsb + a_n^x}{n} \right)
  ^{\frac{n}{x}}$,其中$a_i > 0, i = 1, 2, \dotsc, n$.
\end{ques}
\begin{solu}
  因为$\lim _{x \to 0}a_i^x = 1$,所以原极限是“$1^{\infty}$”型未定式
  \mathmethod 使用\emph{洛必达法则}求极限
  \begin{align*}
    \lim_{x \to 0} \left( \frac{a_1^x + a_2^x + \dotsb + a_n^x}{n} \right)
    ^{\frac{n}{x}} &= \exp \left\{ \lim_{x \to 0}\frac{n}{x} \ln \left(
    \frac{a_1^x + a_2^x + \dotsb + a_n^x}{n} \right)  \right\}\\
    &\mathhint{\text{洛必达}\frac{0}{0}}\\
    &= \exp \left\{ \lim_{x \to 0} n \cdot \frac{a_1^x \ln a_1 + a_2^x \ln a_2
    + \dotsb + a_n^x \ln a_n}{a_1^x + a_2^x + \dotsb + a_n^x} \right\} \\
    &=a_1a_2\dotsi a_n
  \end{align*}
  \mathmethod 凑成第二个重要极限(“$1^{\infty}$”型未定式极限都可以凑成第二个重
  要极限
  \begin{align*}
    \lim_{x \to 0} \left( \frac{a_1^x + a_2^x + \dotsb + a_n^x}{n} \right)
    ^{\frac{n}{x}} &=
    \lim_{x \to 0} \left( 1 + \frac{a_1^x + a_2^x + \dotsb + a_n^x - n}{n}
    \right) ^{\frac{n}{a_1^x + a_2^x + \dotsb + a_n^x - n}
    \cdot \frac{a_1^x + a_2^x + \dotsb + a_n^x - n}{x}}
  \end{align*}
  其中
  \begin{align*}
    \lim_{x \to 0} \frac{a_1^x + a_2^x + \dotsb + a_n^x - n}{x}
    &= \ln (a_1a_2\dotsi a_n)\\
    \mathhint{\lim_{n \to \infty} \left( 1 + \frac{1}{n} \right) ^n = e}\\
    \lim_{x \to 0} \left( 1 + \frac{a_1^x + a_2^x + \dotsb + a_n^x - n}{n}
    \right) ^{\frac{n}{a_1^x + a_2^x + \dotsb + a_n^x - n}} &=e
  \end{align*}
\end{solu}
\end{mathques}

\begin{mathques}{2020-09-06}{极限}
\begin{ques}
  设函数$f(x) = \lim_{n \to \infty} \frac{1 + x}{1 + x^{2n}}$,讨论函数的间断
  点,其结论为(\quad).
  \begin{multichoice}
    \task 不存在间断点
    \task \answer{存在间断点$x = 1$}
    \task 存在间断点$x = 0$
    \task 存在间断点$x = -1$
  \end{multichoice}
\end{ques}
\begin{solu}
\begin{mathideabox}
  函数$f(x)$以$x$为自变量,但是在对$n$求极限的时候,$x$被看做常数,因此应根据
  $x$的不同取值求出对应的极限
\end{mathideabox}
当$|x| < 1$时, $\lim_{n \to \infty} x^{2n} = 0$,所以$f(x) = 1 + x$

当$|x| > 1$时,$\lim_{n \to \infty} \frac{1 + x}{1 + x^{2n}} = 0$

又$f(1) = 1, f(-1) = 0$,则
\[
  f(x) = \lim_{n \to \infty} \frac{1 + x}{1 + x^{2n}} =
  \begin{dcases}
    0,     & x \le -1 \\
    1 + x, & -1 <x < 1 \\
    1,     & x = 1 \\
    0,     & x > 1 \\
  \end{dcases}
\]
故$x = 1$为间断点
\end{solu}
\end{mathques}

\begin{mathques}{2020-09-07}{极限}
\begin{ques}
  设$f(\sin^2x) = \frac{x}{\sin x}$,则$f(x) = $\mathblank.
\end{ques}
\begin{solu}
  \answer{
    $\frac{\arcsin \sqrt{x}}{\sqrt{x}} \matherror{(x > 0)}$
  }\par
  设$u = \sin^2 x$,则$\sin x = \pm \sqrt{u}$.

  当$\sin x = \sqrt{u}$时,$x = \arcsin \sqrt{u}$

  当$\sin x = - \sqrt{u}$时,$\matherror{\sin(-x) = \sqrt{u}, x = -\arcsin
  \sqrt{u}}$.因此
  \begin{align*}
    f(u) &= \frac{\arcsin \sqrt{u}}{\sqrt{u}} \\
    f(x) &= \frac{\arcsin \sqrt{x}}{\sqrt{x}} (x > 0)
  \end{align*}
\end{solu}
\end{mathques}

\begin{mathques}{2020-09-07}{极限}
\begin{ques}
  设$f\left(x + \frac{1}{x}\right) = \frac{x + x^3}{1 + x^4}$,则$f(x) =
  $\mathblank
\end{ques}
\begin{solu}
  \answer{$\frac{x}{x^2 - 2}$}

  \[
    f\left(\matherror{ x + \frac{1}{x}}\right) = \frac{x + x^3}{1 + x^4} =
    \frac{x^2 \left( \matherror{\textstyle\frac{1}{x} + x} \right) }{x^2
    \left( \frac{1}{x^2} + x^2 \right) } = \frac{\frac{1}{x} + x}{\left(
  \matherror{\textstyle\frac{1}{x} + x} \right) ^2 - 2}
  \]
  因此$f(x) = \frac{x}{x^2 - 2}$
\end{solu}
\end{mathques}

\begin{mathques}{2020-09-07}{极限}
\begin{ques}
  求极限$\lim_{x \to 0^+} \frac{2e^{\frac{1}{x}} + e^{-\frac{1}{x}}}
  {e^{\frac{2}{x}} - e^{- \frac{1}{x}}}$.
\end{ques}
\begin{solu}
  由于
  \[
  \lim_{x \to 0^+} \frac{1}{x} = +\infty, \lim_{x \to 0^+} \frac{-1}{x} =
  -\infty
  \]
  因此
  \[
  \lim_{x \to 0^+} e^{\frac{1}{x}} = + \infty, \lim_{x \to 0^+} e^{-\frac{1}
  {x}} = 0
  \]
  则
  \[
    \lim_{x \to 0^+} \frac{2e^{\frac{1}{x}} + e^{-\frac{1}{x}}} {e^{\frac{2}
    {x}} - e^{- \frac{1}{x}}} =
    \lim_{x \to 0^+} \frac{2 + e^{- \frac{2}{x}}}{e^{\frac{1}{x}} -
    e^{-\frac{2}{x}}}
    \mathhint[r]{\text{同除}e^{\frac{1}{x}}}
  \]
\end{solu}
\end{mathques}

\begin{mathques}{2020-09-07}{极限}
\begin{ques}
  求极限$\lim_{x \to 0} \frac{\sin x + x^2 \sin \frac{1}{x}}{(2 + x^2) \ln(1 +
  x)}$.
\end{ques}
\begin{solu}
  \begin{mathideabox}
    所给极限为“$\frac{0}{0}$”型,先进行\emph{等价无穷小代换},再拆分
  \end{mathideabox}
  \begin{align*}
    \lim_{x \to 0} \frac{\sin x + x^2 \sin \frac{1}{x}}{(2 + x^2) \ln(1 + x)}
    &= \lim_{x \to 0} \frac{1}{2 + x^2} \cdot \frac{\sin x + x^2 \sin \frac{1}
    {x}}{x} \mathhint[r]{\text{等价无穷小代换}}\\
    \mathhint[l]{\text{将非零因子}\frac{1}{2 + x^2}\text{单独求极限}}
    &= \frac{1}{2} \lim_{x \to 0} \left( \frac{\sin x}{x} + x \sin \frac{1}{x}
    \right) \\
    &= \frac{1}{2}
  \end{align*}
\end{solu}
\end{mathques}
\end{document}
