\documentclass{ctexart}
\input math.code.tex
% 可以设置复习日期,默认\today
\setdate{2020-11-20}
% 显示所有题目
\mathsetup{
  % show all,
% 设置复习时间点,默认2, 4, 7, 14, 30
  review point = { 3, 10, 30 }
}


\begin{document}

\begin{mathques}{2020-09-05}{极限}
\begin{ques}
  极限$\lim_{n \to \infty} \left( \frac{n+1}{n-2} \right) ^n=$\mathblank.
\end{ques}
\begin{solu}
  \begin{align*}
    \lim_{n \to \infty} \left( \frac{n+1}{n-2} \right) ^n
    &= \lim_{n \to \infty} \frac{\left( 1 + \frac{1}{n} \right) ^n}{\left( 1 -
    \frac{2}{n} \right) ^n}\\
    & \mathhint{\lim_{n \to \infty} \left( 1 + \frac{1}{n} \right) ^n = e}
    \\
    & = \frac{\lim_{n \to \infty} \left( 1 + \frac{1}{n} \right) ^n}{\lim_{n
    \to \infty} \left( 1 - \frac{2}{n} \right) ^n}\\
    & = \frac{e}{e^{-2}} = e^3
  \end{align*}
\end{solu}
\end{mathques}

\begin{mathques}{2020-09-05}{极限}
\begin{ques}
  求极限$\lim_{n \to \infty} (\sqrt{n + \sqrt{n}} - \sqrt{n - \sqrt{n}})$.
\end{ques}
\begin{solu}
  \begin{align*}
    \lim_{n \to \infty} (\sqrt{n + \sqrt{n}} - \sqrt{n - \sqrt{n}})
    &= \lim_{n \to \infty} \frac{(n + \sqrt{n}) - (n - \sqrt{n})}{(n +
    \sqrt{n}) + (n - \sqrt{n})}\\
    &= \lim_{n \to \infty} \frac{2}{\sqrt{1 + \frac{1}{\sqrt{n}}} + \sqrt{1 -
    \frac{1}{\sqrt{n}}}}\\
    &= \frac{2}{\matherror{\sqrt{1} + \sqrt{1}}} = 1
  \end{align*}
\end{solu}
\end{mathques}

\begin{mathques}{2020-09-05}{极限}
\begin{ques}
  设$x_{n + 1} = \sqrt{2 + x_n} (n = 1, 2, \dotsc), x_1 = \sqrt{2}$, 证明
  $\lim_{n \to \infty} x_n$ 存在, 并求$\lim_{n \to \infty} x_n$.
\end{ques}
\begin{solu}
  \mathidea{单调有界数列必有极限}
  \begin{step}{证明单调性(3种方法)}
    \begin{enumerate}
      \item 令$f(x) = \sqrt{2 + x}$ 证明$f(x)$的单调性
      \item 作差$x_{n} - x_{n - 1} = \sqrt{2 + x_{n - 1}} - x_{n - 1} = \frac{2
        + x_{n - 1} - x_{n - 1}^2}{\sqrt{2 + x_{n - 1}} + x_{n - 1}}$\par
        $2 + x_{n - 1} - x_{n - 1}^2 > 2 + x_{n - 1} - 2x_{n - 1} = 2 - x_{n
        - 1} > 0$\par
        \mathidea{方法2需要先进行Step 2证明上界为2才行}
      \item 证明$x_{n + 1} - x_n$与$x_{n} - x_{n - 1}$同号\par
        $x_{n + 1} - x_n = \sqrt{2 + x_n} - \sqrt{2 + x_{n - 1}} = \frac{x_n
        - x_{n - 1}}{\sqrt{2 + x_n} + \sqrt{2 + x_{n - 1}}}$\par
        又$x_2 - x_1 = \sqrt{2 + \sqrt{2}} - \sqrt{2} > 0$,故$x_n$单调
    \end{enumerate}
  \end{step}
  \begin{step}{证明有界性}
    \begin{mathideabox}
      \itshape 当界值难以看出来时,可以先假定有界,得出有极限然后求出根据递
      推公式求出极限得出界值再反过来想怎么证明(见Step 3)
    \end{mathideabox}
    $x_1 = \sqrt{2} < 2$, 设$x_k < 2$, 则$x_{k + 1} = \sqrt{2 + x_k} < 2$,
    故$x_n < 2$有上界
  \end{step}
  \begin{step}{求极限}
    $x_n > 0$, 设$\lim_{n \to \infty} x_n = A > 0$, 则$A = \sqrt{2 + A} \Rightarrow A = 2$
  \end{step}
\end{solu}
\end{mathques}

\begin{mathques}(3){2020-09-22}{极限}
\begin{ques}
  求极限$\lim_{x \to 0^+} \frac{x^x - (\sin x)^x}{x^2 \ln (1 + x)}$.
\end{ques}
\begin{solu}
\begin{align*}
  \text{原式} & = -\matherror{\lim_{x \to 0^+} x^x} \lim_{x \to 0^+}
  \frac{\left( \frac{\sin x}{x} \right) ^x - 1}{x^3}\\
  &\mathhint{e^x - 1 \sim x (x \to 0)}\\
  &= -\lim_{x \to 0^+} \frac{\ln \frac{\sin x}{x}}{x^2}\\
  &= -\lim_{x \to 0^+} \frac{\ln \left( 1 + \frac{\sin x - x}{x} \right) }{x^2}
  \\
  &\mathhint{\ln (1+x) \sim x (x \to 0)}\\
  &= -\lim_{x \to 0^+} \frac{\sin x - x}{x^3}\\
  &= -\lim_{x \to 0^+} \frac{x - \frac{1}{3!} x^3 + o(x^3) - x}{x^3} = \frac{1}
  {6}
\end{align*}
\end{solu}
\end{mathques}

\begin{mathques}(2){2020-09-08}{极限}
\begin{ques}
  求极限$\lim_{x \to 0} \left( \frac{a_1^x + a_2^x + \dotsb + a_n^x}{n} \right)
  ^{\frac{n}{x}}$,其中$a_i > 0, i = 1, 2, \dotsc, n$.
\end{ques}
\begin{solu}
  因为$\lim _{x \to 0}a_i^x = 1$,所以原极限是“$1^{\infty}$”型未定式
  \mathmethod 使用\emph{洛必达法则}求极限
  \begin{align*}
    \lim_{x \to 0} \left( \frac{a_1^x + a_2^x + \dotsb + a_n^x}{n} \right)
    ^{\frac{n}{x}} &= \exp \left\{ \lim_{x \to 0}\frac{n}{x} \ln \left(
    \frac{a_1^x + a_2^x + \dotsb + a_n^x}{n} \right)  \right\}\\
    &\mathhint{\text{洛必达}\frac{0}{0}}\\
    &= \exp \left\{ \lim_{x \to 0} n \cdot \frac{a_1^x \ln a_1 + a_2^x \ln a_2
    + \dotsb + a_n^x \ln a_n}{a_1^x + a_2^x + \dotsb + a_n^x} \right\} \\
    &=a_1a_2\dotsi a_n
  \end{align*}
  \mathmethod 凑成第二个重要极限(“$1^{\infty}$”型未定式极限都可以凑成第二个重
  要极限
  \begin{align*}
    \lim_{x \to 0} \left( \frac{a_1^x + a_2^x + \dotsb + a_n^x}{n} \right)
    ^{\frac{n}{x}} &=
    \lim_{x \to 0} \left( 1 + \frac{a_1^x + a_2^x + \dotsb + a_n^x - n}{n}
    \right) ^{\frac{n}{a_1^x + a_2^x + \dotsb + a_n^x - n}
    \cdot \frac{a_1^x + a_2^x + \dotsb + a_n^x - n}{x}}
  \end{align*}
  其中
  \begin{align*}
    \lim_{x \to 0} \frac{a_1^x + a_2^x + \dotsb + a_n^x - n}{x}
    &= \ln (a_1a_2\dotsi a_n)\\
    \mathhint{\lim_{n \to \infty} \left( 1 + \frac{1}{n} \right) ^n = e}\\
    \lim_{x \to 0} \left( 1 + \frac{a_1^x + a_2^x + \dotsb + a_n^x - n}{n}
    \right) ^{\frac{n}{a_1^x + a_2^x + \dotsb + a_n^x - n}} &=e
  \end{align*}
\end{solu}
\end{mathques}

\begin{mathques}(2){2020-09-08}{极限}
\begin{ques}
  设函数$f(x) = \lim_{n \to \infty} \frac{1 + x}{1 + x^{2n}}$,讨论函数的间断
  点,其结论为(\quad).
  \begin{multichoice}
    \task 不存在间断点
    \task \answer{存在间断点$x = 1$}
    \task 存在间断点$x = 0$
    \task 存在间断点$x = -1$
  \end{multichoice}
\end{ques}
\begin{solu}
\begin{mathideabox}
  函数$f(x)$以$x$为自变量,但是在对$n$求极限的时候,$x$被看做常数,因此应根据
  $x$的不同取值求出对应的极限
\end{mathideabox}
当$|x| < 1$时, $\lim_{n \to \infty} x^{2n} = 0$,所以$f(x) = 1 + x$

当$|x| > 1$时,$\lim_{n \to \infty} \frac{1 + x}{1 + x^{2n}} = 0$

又$f(1) = 1, f(-1) = 0$,则
\[
  f(x) = \lim_{n \to \infty} \frac{1 + x}{1 + x^{2n}} =
  \begin{dcases}
    0,     & x \le -1 \\
    1 + x, & -1 <x < 1 \\
    1,     & x = 1 \\
    0,     & x > 1 \\
  \end{dcases}
\]
故$x = 1$为间断点
\end{solu}
\end{mathques}

\begin{mathques}(2){2020-09-09}{极限}
\begin{ques}
  设$f(\sin^2x) = \frac{x}{\sin x}$,则$f(x) = $\mathblank.
\end{ques}
\begin{solu}
  \answer{
    $\frac{\arcsin \sqrt{x}}{\sqrt{x}} \matherror{(x > 0)}$
  }\par
  设$u = \sin^2 x$,则$\sin x = \pm \sqrt{u}$.

  当$\sin x = \sqrt{u}$时,$x = \arcsin \sqrt{u}$

  当$\sin x = - \sqrt{u}$时,$\matherror{\sin(-x) = \sqrt{u}, x = -\arcsin
  \sqrt{u}}$.因此
  \begin{align*}
    f(u) &= \frac{\arcsin \sqrt{u}}{\sqrt{u}} \\
    f(x) &= \frac{\arcsin \sqrt{x}}{\sqrt{x}} (x > 0)
  \end{align*}
\end{solu}
\end{mathques}

\begin{mathques}{2020-09-07}{极限}
\begin{ques}
  设$f\left(x + \frac{1}{x}\right) = \frac{x + x^3}{1 + x^4}$,则$f(x) =
  $\mathblank
\end{ques}
\begin{solu}
  \answer{$\frac{x}{x^2 - 2}$}

  \[
    f\left(\matherror{ x + \frac{1}{x}}\right) = \frac{x + x^3}{1 + x^4} =
    \frac{x^2 \left( \matherror{\textstyle\frac{1}{x} + x} \right) }{x^2
    \left( \frac{1}{x^2} + x^2 \right) } = \frac{\frac{1}{x} + x}{\left(
  \matherror{\textstyle\frac{1}{x} + x} \right) ^2 - 2}
  \]
  因此$f(x) = \frac{x}{x^2 - 2}$
\end{solu}
\end{mathques}

\begin{mathques}(2){2020-09-09}{极限}
\begin{ques}
  求极限$\lim_{x \to 0^+} \frac{2e^{\frac{1}{x}} + e^{-\frac{1}{x}}}
  {e^{\frac{2}{x}} - e^{- \frac{1}{x}}}$.
\end{ques}
\begin{solu}
  \begin{align*}
    \lim_{x \to 0^+} \frac{2e^{\frac{1}{x}} + e^{-\frac{1}{x}}} {e^{\frac{2}
    {x}} - e^{- \frac{1}{x}}}
    &= \lim_{t \to +\infty} \frac{2e^t + e^{-t}}{e^{2t} - e^{-t}}
    \mathhint[r]{\text{令}t = \frac{1}{x}}\\
    &= \lim_{t \to +\infty} \frac{2 + e^{-2t}}{e^t - e^{-2t}}
    \mathhint[r]{\text{同除}e^t}\\
    &= 0
  \end{align*}
\end{solu}
\end{mathques}

\begin{mathques}(3){2020-09-11}{极限}
\begin{ques}
  求极限$\lim_{x \to 0} \frac{\sin x + x^2 \sin \frac{1}{x}}{(2 + x^2) \ln(1 +
  x)}$.
\end{ques}
\begin{solu}
  \begin{mathideabox}
    所给极限为“$\frac{0}{0}$”型,先进行\emph{等价无穷小代换},再拆分
  \end{mathideabox}
  \begin{align*}
    \lim_{x \to 0} \frac{\sin x + x^2 \sin \frac{1}{x}}{(2 + x^2) \ln(1 + x)}
    &= \lim_{x \to 0} \frac{1}{2 + x^2} \cdot \frac{\sin x + x^2 \sin \frac{1}
    {x}}{x} \mathhint[r]{\text{等价无穷小代换}}\\
    \mathhint[l]{\text{将非零因子}\frac{1}{2 + x^2}\text{单独求极限}}
    &= \frac{1}{2} \lim_{x \to 0} \left( \frac{\sin x}{x} + x \sin \frac{1}{x}
    \right) \\
    &= \frac{1}{2}
  \end{align*}
\end{solu}
\end{mathques}

\begin{mathques}{2020-09-11}{一元微分}
\begin{ques}
  设函数$f(x)$在$x = 0$处连续,且$\lim_{x \to 0} \frac{f(x^2)}{x^2} = 1$,则
  (\quad)
  \begin{multichoice}
    \task $f(0) = 0$且$f'_-(0)$存在
    \task $f(0) = 1$且$f'_-(0)$存在
    \task \answer{$f(0) = 0$且$f'_+(0)$存在}
    \task $f(0) = 1$且$f'_+(0)$存在
  \end{multichoice}
\end{ques}
\begin{solu}
  因为$f(x)$在$x = 0$处连续,且$\lim_{x \to 0} \frac{f(x^2)}{x^2} = 1$

  所以$\lim_{x \to 0} f(x^2) = 0$,即$f(0) = 0$

  从而有
  \begin{align*}
    \lim_{x \to 0} \frac{f(x^2)}{x^2}
    &= \matherror{\lim_{x \to 0} \frac{f(x^2) - f(0)}{x^2 - 0}}\\
    \mathhint[l]{t = x^2} &= \matherror{\lim_{t \to \color{red} 0^+}
    \frac{f(t) - f(0)}{t - 0}}\\
    &= f'_+(0)
  \end{align*}
\end{solu}
\end{mathques}

\begin{mathques}{2020-09-11}{一元微分}
\begin{ques}
  设函数$f(x) = \abs{x^3 - 1}\varphi (x)$,其中$\varphi (x) $在$x = 1$处连续,
  则$\varphi (1) = 0$是$f(x)$在$x = 1$处可导的(\quad)
  \begin{multichoice}
    \task \answer{充分必要条件}
    \task 充分但非必要条件
    \task 必要但非充分条件
    \task 既非充分也非必要条件
  \end{multichoice}
\end{ques}
\begin{solu}
  \mathidea{$\varphi(1) = 0 \Longrightarrow f(x)$在$x = 1$处可导}\par
  由$\varphi(1) = 0$可得$\mathhint{f(1) = 0}$
  \begin{align*}
    f'_+(1) &= \lim_{x \to 1^+} \frac{f(x) - f(1)}{x - 1}
    = \lim_{x \to 1^+} \frac{\abs{x^3 - 1}\varphi(x)}{x - 1}
    = \lim_{x \to 1^+} (x^2 + x + 1)\varphi(x) = 0\\
    f'_-(1) &= \lim_{x \to 1^-} \frac{f(x) - f(1)}{x - 1}
    = \lim_{x \to 1^-} \frac{\abs{x^3 - 1}\varphi(x)}{x - 1}
    = \lim_{x \to 1^-} (x^2 + x + 1)\varphi(x) = 0
  \end{align*}
  即$f'_+(1) = f'_-(1) = 0$,则$f'(1) = 0$

  \mathidea{$\varphi(1) = 0 \Longleftarrow f(x)$在$x = 1$处可导}\par
  设$f(x)$在$x = 1$处可导,因为$f(1) = 0$,所以
  \begin{align*}
    f'_+(1) &= \lim_{x \to 1^+} \frac{f(x) - f(1)}{x - 1}
    = \lim_{x \to 1^+} \frac{\abs{x^3 - 1}\varphi(x)}{x - 1}
    = \lim_{x \to 1^+} (x^2 + x + 1)\varphi(x) = 3\varphi(1)\\
    f'_-(1) &= \lim_{x \to 1^-} \frac{f(x) - f(1)}{x - 1}
    = \lim_{x \to 1^-} \frac{\abs{x^3 - 1}\varphi(x)}{x - 1}
    = \lim_{x \to 1^-} (x^2 + x + 1)\varphi(x) = -3\varphi(1)
  \end{align*}
  由$f'_+(1) = f'_-(1) = 0$得,$3\varphi(1) = -3\varphi(1)$,即$\varphi(1) = 0$
\end{solu}
\end{mathques}

\begin{mathques}{2020-09-11}{一元微分}
\begin{ques}
  设函数$y = f(x)$由方程$y = x\ln y$所确定,求$\dv{y}{x}$.
\end{ques}
\begin{solu}
\mathmethod 方程两边同时对$x$求导,得$y' = \ln y + x\cdot \frac{1}{y} \cdot
y'$,解得$y' = \frac{y \ln y}{y - x}$.
\mathmethod 令$F(x, y) = y - x\ln y$,则$F'_x = - \ln y, F'_y = 1 - \frac{x}{y}
$,故$\dv{y}{x} = - \frac{F'_x}{F'_y} = \frac{y\ln y}{y - x}$.
\end{solu}
\end{mathques}

\begin{mathques}(2){2020-09-13}{一元微分}
\begin{ques}
  已知$f'(x) = Ae^x$($A$为正常数),求$f(x)$的反函数的二阶导数.
\end{ques}
\begin{solu}
  \begin{mathideabox}[反函数定理]
    如果从$\mathbb{R}^n$的一个开集$\mathrm{U}$到$\mathbb{R}^n$的连续可微函数
    $F$的全微分在点$p$可逆($F$在点p的雅可比行列式不为零),则$F$在点$p$附近具
    有反函数,且
    \[
      J_{F^{-1}}(F(p)) = [J_F(p)]^{-1}
    \]
    其中$[\cdot]^{-1}$表示逆矩阵,$J_G(q)$是函数$G$在点$q$的雅可比矩阵。

    也可由链式法则推出:设$G,H$是两个函数,分别在$H(p)$和$p$处有全导数,则
    \[
      J_{G\circ H}(p) = J_G(H(p))\cdot J_H(p)
    \]
    设$G$为$F$,$H$为$F^{-1}$,$G\circ H$则为恒等函数,其雅可比矩阵也是单位矩
    阵。

    当$y$为$x$的一元函数时,其雅可比矩阵为$[\dd y / \dd x]$,即
    \[
      \dv{x}{y} = \frac{1}{\dv{y}{x}}
    \]
  \end{mathideabox}
  设$y = f(x)$,则$\dv{x}{y} = \frac{1}{f'(x)}$
  \begin{align*}
    \dv[2]{y}{x} &= \dv{y}(\dv{x}{y})\\
    &= \matherror{\dv{x}(\frac{1}{f'(x)}) \cdot \dv{x}{y}}\\
    &= \matherror{-\frac{f''(x)}{\qty(f'(x))^2} \cdot \frac{1}{f'(x)}}\\
    &= -\frac{f''(x)}{\qty(f'(x))^3} = - \frac{Ae^x}{(Ae^x)^3} = -\frac{1}
    {A^2e^{2x}}
  \end{align*}
\end{solu}
\end{mathques}

\begin{mathques}(2){2020-09-13}{一元微分}
\begin{ques}
  设可导函数$y = f(x)$有反函数$g(x)$,且$f(a) = 3, f'(a) = 1, f''(a) = 2$,求
  $g''(3)$.
\end{ques}
\begin{solu}
  \mathmethod
  由链式求导法则得$\qty(g \circ f(x))' = g'\qty(f(x)) \cdot f'(x)$,即
  \[
    \matherror{f'(x) \cdot g'(f(x)) = 1}
  \]
  两边同时对$x$求导,有
  \[
    f''(x)g'(f(x)) + \qty(f'(x))^2 \cdot g''(f(x)) = 0
  \]
  两边同乘$f'(x)$
  \[
    f''(x) + \qty(f'(x))^3 \cdot g''(f(x)) = 0
  \]
  将$x = a$代入得
  \[
    f''(a) + \qty(f'(a))^3 \cdot g''(f(a)) = 0
  \]
  即
  \[
    2 + g''(3) = 0
  \]
  则$g''(3) = -2$
  \mathmethod 公式法
  \[
    g''(3) = - \frac{f''(a)}{\qty(f(a))^3} = -2
  \]
  \begin{mathideabox}[公式推导过程]
    已知
    $\left\{\begin{aligned}
      \eval{y}_{x = a} &= 3\\
      \eval{\dv{y}{x}}_{x = a} &= 1\\
      \eval{\dv[2]{y}{x}}_{x = a} &= 2
    \end{aligned}\right.$
    , 求$\eval{\dv[2]{x}{y}}_{y = 3}$

    \begin{align*}
      \dv[2]{x}{y} &= \dv{y}(\dv{x}{y}) = \dv{x}(\dv{x}{y}) \cdot \dv{x}{y} \\
      &= \dv{x}(\frac{1}{\dv{y}{x}}) \cdot \frac{1}{\dv{y}{x}} \\
      &= - \frac{\dv{x}(\dv{y}{x})}{\qty(\dv{y}{x})^2} \cdot \frac{1}{\dv{y}{x}}\\
      &= - \frac{\dv[2]{y}{x}}{\qty(\dv{y}{x})^3}
    \end{align*}
    即$\eval{\dv[2]{x}{y}}_{y = 3} = \eval{- \frac{\dv[2]{y}{x}}{\qty(\dv{y}
    {x})^3}}_{x = a}$
  \end{mathideabox}
\end{solu}
\end{mathques}

\begin{mathques}{2020-09-13}{一元微分}
\begin{ques}
  设$f(x) =
  \begin{dcases}
    x^2 \sin \frac{\pi}{x}, & x < 0, \\
    A, & x = 0, \\
    ax^2 + b, & x > 0,\\
  \end{dcases}
  $求常数$A, a, b$的值,使$f(x)$在$x = 0$处可导,并求$f'(0)$.
\end{ques}
\begin{solu}
  \begin{mathideabox}[可导必然连续(反之不一定)]
    设函数$y = f(x)$上一点$x_0$,函数在这一点可导,即$f'(x_0) = \lim_{\Delta x
    \to 0} \frac{\Delta y}{\Delta x}存在$,其中
    \[
      \Delta y = f(x_0 + \Delta x) - f(x_0)
    \]
    则
    \begin{align*}
      \lim_{\Delta x \to 0} \Delta y &= \lim_{\Delta x \to 0} \qty(\frac{\Delta
      y}{\Delta x}\cdot \Delta x)\\
      &= f'(x_0) \cdot 0 = 0
    \end{align*}
    即函数$f(x)$在$x_0$处连续
  \end{mathideabox}
  \begin{mathideabox}[导数存在必然有左右导数相等]
    由极限的性质可知:
    \[
      \lim_{\Delta x \to 0} \frac{f(x_0 + \Delta x) - f(x_0)}{\Delta x}
      =\lim_{\Delta x \to 0^-} \frac{f(x_0 + \Delta x) - f(x_0)}{\Delta x}
      =\lim_{\Delta x \to 0^+} \frac{f(x_0 + \Delta x) - f(x_0)}{\Delta x}
    \]
  \end{mathideabox}
  可导必然连续,即
  \[
    \lim_{x \to 0^-} x^2 \frac{\pi}{x} = \lim_{x \to 0^+} (ax^2 + b) = A
  \]
  则$A = b = 0$.
  又
  \begin{align*}
    f'_-(0) &= \lim_{x \to 0^-} \frac{x^2 \sin \frac{\pi}{x} - 0}{x - 0} = 0\\
    f'_+(0) &= \lim_{x \to 0^+} \frac{ax^2 - 0}{x - 0} = 0
  \end{align*}
  所以,$a$可以为任意常数,且$f'(0) = 0$
\end{solu}
\end{mathques}

\begin{mathques}(2){2020-09-18}{一元微分}
\begin{ques}
  设函数$y = f(x)$由$
  \begin{dcases}
    x = \ln(1 + t^2) + 1,\\
    y = 2 \arctan t - (t + 1)^2,\\
  \end{dcases}
  $确定,求$\dv{y}{x}, \dv[2]{y}{x}$
\end{ques}
\begin{solu}
  \begin{align*}
    \dv{y}{t} &= \frac{2}{\matherror{1 + t^2}} - 2(t + 1)\\
    \dv{x}{t} &= \frac{2t}{1 + t^2}
  \end{align*}
  由链式法则,$\dv{y}{t} = \dv{y}{x}\cdot\dv{x}{t}$,则
  \begin{align*}
    \dv{y}{x} &= \frac{\dv{y}{t}}{\dv{x}{t}}
    = \frac{\frac{2}{1 + t^2} - 2(t + 1)}{\frac{2t}{1 + t^2}}
    = -(t^2 + t + 1)\\
    \dv[2]{y}{x} &= \dv{x}(\dv{y}{x})\\
    &= \dv{x}(-(t^2 + t + 1))\\
    &= \dv{t}(-(t^2 + t + 1)) \cdot \dv{t}{x}\\
    &= -\frac{1 + 2t}{\dv{x}{t}}\\
    &= -\frac{(1 + 2t)(1 + t^2)}{2t}
  \end{align*}
\end{solu}
\end{mathques}

\begin{mathques}(2){2020-09-15}{一元微分}
\begin{ques}
  设$f(x)$满足$f(0) = 0$,且$f'(0)$存在,求$\lim_{x \to 0} \frac{f(1 -
  \sqrt{\cos x})}{\ln(1 - x\sin x)}$
\end{ques}
\begin{solu}
  \begin{align*}
    \lim_{x \to 0} \frac{f(1 - \sqrt{\cos x})}{\ln(1 - x\sin x)}
    &=\lim_{x \to 0} \frac{f(1 - \sqrt{\cos x}) - f(0)}{(1 - x\sin x) - 0}
    \cdot \lim_{x \to 0} \frac{1 - \sqrt{\cos x}}{\ln(1 - x\sin x)}\\
    &=f'(0) \lim_{x \to 0} \frac{1 - \sqrt{\cos x}}{\ln(1 - x\sin x)}\\
    \mathidea{等价无穷小替换}&=f'(0)\lim_{x \to 0} \frac{1 - \cos x}{-x\sin x}
    \cdot \frac{1}{1 + \sqrt{\cos x}}\\
    &= -\frac{1}{2}f'(0)\lim_{x \to 0} \frac{1 - \cos x}{x^2}\\
    &= -\frac{1}{4}f'(0).
  \end{align*}
\end{solu}
\end{mathques}

\begin{mathques}{2020-09-12}{一元微分}
\begin{ques}
  设函数$f(x)$在$x = 0$处连续,且$\lim_{x \to 0} \frac{f(x)}{x}$存在,则(\quad)
  \begin{multichoice}
    \task $f(0) \neq 0$,但$f'(0)$可能不存在
    \task $f(0) = 0$,但$f'(0)$可能不存在
    \task \answer{$f'(0)$存在,但$f'(0)$不一定等于零}
    \task $f'(0)$存在,且必定有$f'(0) = 0$
  \end{multichoice}
\end{ques}
\begin{solu}
  \begin{mathideabox}[连续的定义]
    函数$f(x)$在点$p$处连续当且仅当$\lim_{x \to p} f(x) = f(p)$ (隐含着极限必
    须存在)
  \end{mathideabox}
  \begin{mathideabox}[重要结论]
    若$f(x)$在$x = 0$处连续,且$\lim_{x \to 0} \frac{f(x)}{x} = A$,则
    \begin{align*}
      f(0) &= 0\\
      f'(0) &= A
    \end{align*}
  \end{mathideabox}
  $f(x)$在$x = 0$处连续,则$\lim_{x \to 0} f(x)$存在且$\lim_{x \to 0} f(x) =
  f(0)$

  由$\lim_{x \to 0} \frac{f(x)}{x}$存在且分母极限为$0$可知,分子极限也为$0$,
  即$f(0) = 0$

  又
  \[
    f'(0) = \lim_{x \to 0} \frac{f(x) - f(0)}{x - 0} = \lim_{x \to 0}
    \frac{f(x)}{x}
  \]
  即$f'(0)$存在
\end{solu}
\end{mathques}

\begin{mathques}{2020-09-12}{一元微分}
\begin{ques}
  设$f(x) = x^a \abs{x}$,$a$为正整数,则函数$f(x)$在点$x = 0$处(\quad)
  \begin{multichoice}
    \task 不存在极限
    \task 存在极限,但不连续
    \task 连续但不可导
    \task \answer{可导}
  \end{multichoice}
\end{ques}
\begin{solu}
  $\lim_{x \to 0} f(x) = 0 = f(0)$,故$f(x)$在$x = 0$处连续
  \begin{align*}
    f'_+(0) &= \lim_{x \to 0^+} \frac{f(x) - f(0)}{x - 0} = \lim_{x \to 0^+}
    \frac{x^a \abs{x}}{x} = \matherror{\lim_{x \to 0^+} \frac{x^{a + 1}}{x} =
    0}\\
    f'_-(0) &= \lim_{x \to 0^-} \frac{f(x) - f(0)}{x - 0} = \lim_{x \to 0^-}
    \frac{x^a \abs{x}}{x} = \lim_{x \to 0^-} \frac{-x^{a + 1}}{x} = 0
  \end{align*}
  即$f'_+(0) = f'_-(0) = 0$,$f(x)$在点$x = 0$处可导,且导数为$0$.
\end{solu}
\end{mathques}

\begin{mathques}{2020-09-12}{一元微分}
\begin{ques}
  若$y = f(x)$可导,则当$\Delta x \to 0$时,$\Delta y - \dd y$为$\Delta x$的
  (\quad)
  \begin{multichoice}
    \task \answer{高阶无穷小}
    \task 低阶无穷小
    \task 同阶但不等价无穷小
    \task 等阶无穷小
  \end{multichoice}
\end{ques}
\begin{solu}
  \begin{mathideabox}[微分定义]
    设函数$y = f(x)$在区间$\mathcal{I}$有定义。对于$\mathcal{I}$内的一点$x_0$,
    当$x_0$变动到附近的$x_0 + \Delta x$(也在此区间)时,如果函数的增量$\Delta
    y = f\qty(x_0 + \Delta x - f(x_0))$可以表示为
    \[
      \Delta y = A\Delta x + o(\Delta x)
    \]
    其中$A$是不依赖于$\Delta x$的常数,$o(\Delta x)$是比$\Delta x$高阶的无穷小,
    则称函数$f(x)$在点$x_0$是可微的.

    $A\Delta x$称作函数在点$x_0$对应与自变量增量$\Delta x$的微分,记做$\dd
    y$,即$\dd y = A\Delta x$

    把自变量$x$的增量$\Delta x$称作自变量的微分,记做$\dd x$,即$\dd x =
    \Delta x$
  \end{mathideabox}
  由微分的定义可知$\Delta y - \dd y = o(\Delta x)$为$\Delta x$的高阶无穷小
\end{solu}
\end{mathques}

\begin{mathques}{2020-09-12}{一元微分}
\begin{ques}
设函数
\[
  f(x) =
  \begin{dcases}
    x^3\sin \frac{1}{x}, & x \neq 0,\\
    0, & x= 0,\\
  \end{dcases}
\]
讨论$f(x)$在$x = 0$处的可导性以及$f'(x)$在$x = 0$处的连续性.
\end{ques}
\begin{solu}
  \begin{mathideabox}
    证明$f(x)$在$x_0$处可导,即证明
    \begin{align*}
      f'(x_0) &= \lim_{\Delta x \to 0} \frac{\Delta y}{\Delta x}
      = \lim_{\Delta x \to 0} \frac{f(x_0 + \Delta x) - f(x_0)}{\Delta x}
      \\
      \text{或}f'(x_0) &= \lim_{x \to x_0} \frac{f(x) - f(x_0)}{x - x_0}
    \end{align*}
    存在
    \par\vspace{1em}
    证明$f(x)$在$x_0$处连续,即证明
    \[
      \lim_{x \to x_0} f(x) = f(x_0)
    \]
  \end{mathideabox}

  \matherror{\text{当}x = 0\text{时}},有
  \[
    f'(0) = \lim_{x \to 0} \frac{x^3 \sin \frac{1}{x} - 0}{x - 0} = 0,
  \]
  函数$f(x)$在$x = 0$处可导.

  \matherror{\text{当}x \neq 0\text{时}},有
  \[
    f'(x) = 3x^2 \sin \frac{1}{x} - x\cos \frac{1}{x}
  \]
  因为
  \[
    \lim_{x \to 0} f'(x) = \lim_{x \to 0} \qty(3x^2 \sin \frac{1}{x} - x\cos
    \frac{1}{x}) = 0 = f'(0),
  \]
  故函数f'(x)在$x = 0$处连续
\end{solu}
\end{mathques}

\begin{mathques}{2020-09-16}{一元微分}[最后忘记给$a$加平方]
\begin{ques}
已知曲线$y = x^3 - 3a^2x + b$与$x$轴相切,则$b^2$可通过$a$表示为
$b^2=$\mathblank
\end{ques}
\begin{solu}
  \answer{$4a^6$}

  因为曲线$y = x^3 - 3a^2 x + b$与$x$轴相切,所以,在切点处有
  \[
  y' = 3x^2 - 3a^2 = 0.
  \]
  由此可解出切点横坐标$x = a$或$x = -a$,且在切点处有
  \[
    y(a) = a^3 - 3a^3 + b = 0 \qq{或} y(-a) = -a^3 + 3 a^3 + b = 0
  \]
  则$b = \pm 2a^3$,故$b^2 = 4a^6$
\end{solu}
\end{mathques}

\begin{mathques}{2020-09-16}{一元微分}[没有考虑导数的分母也会跟着改变正负号]
\begin{ques}
设$y = \frac{x^3 + 4}{x^2}$,求函数的增减区间及极值
\end{ques}
\begin{solu}
  $\matherror{y' = 1 - \frac{8}{x^3}}$,故驻点为$x = 2$,又
  \[
  \begin{NiceMatrix}[columns-width=2cm]
  \toprule
  x & (-\infty, 0) & (0, 2) & 2 & (2, +\infty)\\
  \midrule
  y' & + & - & 0 & + \\
  y & \nearrow & \searrow & 3 & \nearrow\\
  \bottomrule
  \end{NiceMatrix}
  \]
  所以,$(-\infty, 0)$以及$(2, +\infty)$为增区间,$(0, 2)$为减区间,$x = 2$为
  极小值点,极小值为$y = 3$.
\end{solu}
\end{mathques}

\begin{mathques}{2020-09-16}{一元微分}[算到后面忘记开根号]
\begin{ques}
给定曲线$y = \frac{1}{x^2}$
\begin{queslist}
  \item 求曲线在横坐标为$x_0$的点处的切线方程
  \item 求曲线的切线被两坐标轴所截线段的最短长度
\end{queslist}
\end{ques}
\begin{solu}
\begin{solulist}
  \item 由$y' = -\frac{2}{x^3}$可知曲线$y = \frac{1}{x^2}$在横坐标$x_0$处的切
    线方程为
    \[
      y - \frac{1}{x_0^2} = - \frac{2}{x_0^3}(x - x_0)
    \]
  \item 将$x = 0, y = 0$代入$x = x_0$处切线方程可得该切线在$x$轴,$y$轴上的截
    距分别为
    \[
    X = \frac{3}{2}x_0 \qc Y = \frac{3}{x_0^2}.
    \]
    设该切线被两坐标轴所截线段长度为$L$,则$L^2 = X^2 + Y^2 = \frac{9}{4}x_0^2
    + \frac{9}{x_0^4}.$

    令$\dv{L^2}{x_0} = \frac{9}{2}x_0 - \frac{36}{x_0^5} = 0$,得驻点$x_0 =
    \pm \sqrt{2}.$

    又$\dv[2]{L^2}{x_0} = \frac{9}{2} + \frac{180}{x_0^6}$,显然$\eval{\dv[2]
    {L^2}{x_0}}_{x_0 = \pm \sqrt{2}} > 0$,由此可知,$L^2$在$x_0 = \pm \sqrt{2}
    $处取得极小值,即最小值,$L_{\min}^2 = \frac{27}{4}$

    故$L_{\min} = \frac{3}{2}\sqrt{3}.$
\end{solulist}
\end{solu}
\end{mathques}

\begin{mathques}(2){2020-09-23}{一元微分}[题目问的是最小值不是极小值]
\begin{ques}
  设$a > 1, f(t) = a^t - at$在$(-\infty, +\infty)$内的驻点为$t(a)$,问$a$为何
  值时,$t(a)$最小,并求出最小值.
\end{ques}
\begin{solu}
  \begin{mathideabox}
    先求驻点函数$t(a)$,然后再求$t(a)$的最小值和最小值点
  \end{mathideabox}
  令$f'(t) = a^t\ln a - a = 0$,得唯一驻点函数$t(a) = 1 - \frac{\ln\ln a}{\ln
  a}.$

  令$t'(a) = - \frac{\frac{1}{a} - \frac{1}{a}\ln \ln a}{(\ln a)^2} = 0$,得唯
  一驻点$a = e^e.$

  当$a < e^e$时,$t'(a) < 0$;当$a > e^e$时,$t'(a) > 0$,所以,$t(e^e) = 1 -
  \frac{1}{e}$为极小值,也是最小值.
\end{solu}
\end{mathques}

\begin{mathques}{2020-09-17}{一元微分}[没有直接两边取对数,最后没有化到最简]
\begin{ques}
  设$y = y(x)$由方程$x = y^y$确定,则$\dd y = $\mathblank
\end{ques}
\begin{solu}
  \answer{$\frac{1}{x(1 + \ln y)}\dd x$}

  两边取对数得,$\ln x = y\ln y,$

  两边关于$x$求导,得$\frac{1}{x} = (1 + \ln y) y'$

  则$y' = \frac{1}{x (1 + \ln y)}$

  即$\dd y = y' \dd x = \frac{1}{x(1 + \ln y)}\dd x$
\end{solu}
\end{mathques}

\begin{mathques}{2020-09-17}{一元微分}[漏抄了]
\begin{ques}
  设$y = e^{x^2}$,求$\dv{y}{x}, \dv{y}{(x^2)}, \dv[2]{y}{x}.$
\end{ques}
\begin{solu}
  \begin{mathideabox}[区分微分符号]
    $\dd[n]x = \qty(\dd x)^n$是“微分的幂”,$\dd (x^n) = nx^{n - 1}\dd x$是“幂
    的微分”
  \end{mathideabox}
  \begin{align*}
    \dv{y}{x} &= 2xe^{x^2},\\
    \dv{y}{(x^2)} &= \matherror{\dv{y}{x}\cdot \dv{x}{(x^2)}} = 2xe^{x^2}\cdot
    \frac{1}{2x} = e^{x^2},\\
    \dv[2]{y}{x} &= 2(1 + 2x^2)e^{x^2}.
  \end{align*}
\end{solu}
\end{mathques}

\begin{mathques}{2020-09-17}{一元微分}[计算错误,没有化简后再求导]
\begin{ques}
  设$y = \ln \sqrt{\frac{1 - x}{1 + x^2}}$,求$\eval{y''}_{x = 0}.$
\end{ques}
\begin{solu}
  \begin{align*}
    y &= \matherror{\ln \sqrt{\frac{1 - x}{1 + x^2}} = \frac{1}{2} \qty(\ln (1
    - x) - \ln(1 + x^2))},\\
    y' &= \frac{1}{2}\qty(\frac{-1}{1 - x} - \frac{2x}{1 + x^2})
    = - \frac{1}{2}\qty(\frac{1}{1 - x} + \frac{2x}{1 + x^2}),\\
    y'' &= -\frac{1}{2}\qty(\frac{1}{(1 - x)^2} + 2\cdot \frac{1 - x^2}{(1 +
    x^2)^2}),\\
    \eval{y''}_{x = 0} &= -\frac{3}{2}.
  \end{align*}
\end{solu}
\end{mathques}

\begin{mathques}(2){2020-09-27}{中值定理}[没有注意被积函数的导数与欲证部分的关系+
  积分中值定理不熟悉]
\begin{ques}
  设$f(x)$在$[0, 1]$上连续,在$(0, 1)$内可导,且$f(1) = k\int_{0}^{\frac{1}{k}}
  x e^{1-x} f(x) \dd x (k > 1).$证明至少存在一点$\xi \in (0, 1)$,使得$f'(\xi)
  = (1 - \xi^{-1}) f(\xi)$.
\end{ques}
\begin{solu}
\begin{mathideabox}
  注意到被积函数$x e^{1 - x} f(x)$的导数
  \[
    \qty (xe^{1-x} f(x))' = xe^{1-x} \qty(f'(x) - (1 - x^{-1})f(x))
  \]
  中含有$f'(x) - (1 - x^{-1})f(x),$这正是欲证结论中的一部分
\end{mathideabox}
\mathproof 令$F(x) = xe^{1-x} f(x).$由积分中值定理可知
\begin{align*}
  f(1) &= k \int_{0}^{\frac{1}{k}} x e^{1-x} f(x) \dd x \\
  &= \eta e^{1 - \eta} f(\eta) \\
  &= \matherror{1} \cdot e^{1 - 1} f(1)
\end{align*}
其中$\eta \in \qty(0, \frac{1}{k}) \subset \matherror{(0, 1)}$,
又$F(x) = xe^{1 - x}f(x)$在$[\eta, 1]$上连续,在$(\eta, 1)$内可导,且$F(\eta)
= F(1),$则$F(x) = xe^{1 - x}f(x)$在$[\eta, 1]$上满足罗尔定理的条件

即存在$\xi \in (\eta, 1) \subset (0, 1),$使得$F'(\xi) = 0$, 即$f'(\xi) = (1 -
\xi^{-1}) f(\xi).$
\end{solu}
\end{mathques}

\begin{mathques}(4){2020-10-31}{中值定理}[费马定理+脱帽性不熟悉]
\begin{ques}
  设$f(x)$在$[a, b]$上可导,若$f'_+(a) \neq f'_-(b)$,证明对于任意的介于
  $f'_+(a)$与$f'_-(b)$之间的$\mu$,存在$\xi \in (a, b)$,使得$f'(\xi) = \mu$.
\end{ques}
\begin{solu}
  \mathproof 因$f'_+(a) \neq f'_-(b)$,不妨设$f'_+(a) < f'_-(b)$. 并设$F(x) =
  f(x) - \mu x$,则函数$F(x)$在$[a, b]$上可导,且$F'_+(a) = f'_+(a) - \mu < 0,
  F'_(b) = f'_-(b) - \mu > 0$,于是
  \begin{align*}
    F'_+(a) &= \lim_{x \to a^+} \frac{F(x) - F(a)}{x - a} < 0,\\
    F'_+(b) &= \lim_{x \to b^-} \frac{F(x) - F(b)}{x - b} > 0,
  \end{align*}
  根据\matherror*{极限的保号性},知:
  {\parindent=2\ccwd\par
    在点$x=a$的某个右邻域内,$\frac{F(x) - F(a)}{x - a} < 0$,即$F(x) < F(a)$;
    \vspace{1em}

    在点$x=b$的某个左邻域内,$\frac{F(x) - F(b)}{x - b} > 0$,即$F(x) > F(a)$.
  }

  故$F(a)$和$F(b)$均不是函数$F(x)$在$[a, b]$上的最小值,又因$F(x)$一定可以取得
  最小值,则其最小值必在$(a, b)$内取到,设函数$F(x)$在$(a, b)$内的最小值点是
  $\xi$,根据费马定理,得$F'(\xi) = 0$,即$f'(\xi) = \mu$.
\end{solu}
\end{mathques}

\begin{mathques}(2){2020-09-27}{中值定理}[没有联系第一小问+拉格朗日中值定理不熟悉]
\begin{ques}
  已知$f(x)$在$[0, 1]$上连续,在$(0, 1)$内可导,且$f(0) = 0, f(1) = 1$. 证明:
  \begin{queslist}
    \item 存在$\xi \in (0, 1)$,使得$f(\xi) = 1 - \xi$;
    \item 存在$\eta , \tau \in (0, 1), \eta \neq \tau$,使得$f'(\eta)f'(\tau)
      = 1.$
  \end{queslist}
\end{ques}
\begin{solu}
\begin{solulist}
  \item 令$F(x) = f(x) - 1 + x$, 可得
    \begin{align*}
      F(0) &= f(0) - 1 + 0 = -1 < 0,\\
      F(1) &= f(1) - 1 + 1 = \phantom{-}1 > 0,
    \end{align*}
    由零点定理可知,存在$\xi \in (0, 1)$,使得$F(\xi) = 0$,即$f(\xi) = 1 -
    \xi$.
  \item 用$\xi$将$[0, 1]$划分为$[0, \xi], [\xi, 1]$,再用拉格朗日中值定理
    \begin{align*}
      f(\xi) - f(0) &= f'(\eta)(\xi - 0), & &\eta \in (0, \xi),\\
      f(1) - f(\xi) &= f'(\tau)(1 - \xi), & &\tau \in (\xi, 1),
    \end{align*}
    则
    \begin{align*}
      f'(\eta) &= \frac{f(\xi) - f(0)}{\xi} = \frac{1 - \xi}{\xi}, \\
      f'(\tau) &= \frac{f(1) - f(\xi)}{ 1 - \xi } = \frac{\xi}{1 - \xi},
    \end{align*}
    故$f'(\eta)f'(\tau) = 1$.
\end{solulist}
\end{solu}
\end{mathques}

\begin{mathques}(2){2020-10-01}{零点问题与微分不等式}[没有想到用取对数后的式子证
  明]
\begin{ques}
设$0 < x < 1$,求证$xe^{-x} > \frac{1}{x} e^{-\frac{1}{x}}$.
\end{ques}
\begin{solu}
  \mathproof 要证明$xe^{-x} > \frac{1}{x} e^{-\frac{1}{x}}$,只需证$\ln x - x
  > -\ln x - \frac{1}{x}$,即$2\ln x -x + \frac{1}{x} > 0$.

  设$f(x) = 1\ln x - x + \frac{1}{x}, x \in (0, 1)$,则$f'(x) = \frac{-(x -
  1)^2}{x^2} < 0$,故$f(x)$在$(0, 1)$上单调递减

  则$f(x) > f(1) = 0$.
\end{solu}
\end{mathques}

\begin{mathques}(2){2020-10-01}{零点问题与微分不等式}[计算繁琐]
\begin{ques}
  证明:$\qty(\ln \frac{1+x}{x} - \frac{1}{1+x})^2 < \frac{1}{x(1+x)^2} (x > 0)
  .$
\end{ques}
\begin{solu}
  只需证明当$x > 0$时,$\abs{\ln \frac{1+x}{x} - \frac{1}{1+x}} - \frac{1}
  {\sqrt{x}(1 + x)} < 0$.

  令
  \[
    f(x) = \ln \frac{1 + x}{x} - \frac{1}{1 + x} = \ln (1 + x) - \ln x -
    \frac{1}{1 + x},
  \]

  则
  \begin{align*}
    f'(x) &= \frac{1}{1+x} - \frac{1}{x} + \frac{1}{(1+x)^2} \\
    &= \frac{(1+x)x - (1+x)^2 + x}{x(1 + x)^2} \\
    &= \frac{-1}{x(1+x)^2} < 0,
  \end{align*}
  又$\matherror{\lim_{x \to +\infty} f(x) = 0}$,所以$f(x) > 0$.
  令
  \[
    g(x) = \ln \frac{x + 1}{x} - \frac{1}{1 + x} - \frac{1}{\sqrt{x}(1 + x)},
  \]
  则
  \begin{align*}
    g'(x) &= \frac{1}{1 + x} - \frac{1}{x} + \frac{1}{(1+x)^2} + \frac{\frac{1}
    {2\sqrt{x}}\cdot (1 + x) + \sqrt{x}}{x(1 + x)^2} \\
    &= \frac{-2\sqrt{x} + 1 + x + 2x}{2x^{\frac{3}{2}} (1 + x)^2} \\
    &= \frac{1 + 3x - 2\sqrt{x}}{2x^{\frac{3}{2}} (1 + x)^2}.
  \end{align*}
  再令$h(x) = 1 + 3x - 2\sqrt{x}$,则$h'(x) = 3 - \frac{1}{\sqrt{x}}$,令$h'(x)
  = 0$得$x = \frac{1}{9}$为唯一极小值点,也就是最小值点,且$h_{\min} = \frac{2}
  {3} > 0$,故$h(x) > 0$,则$g'(x) > 0$,即$g(x)$单调增加

  又$\lim_{x \to +\infty} f(x) = 0$且$\lim_{x \to +\infty} - \frac{1}
  {\sqrt{x}(1 + x)} = 0$故$g(x) < 0$,证毕.
\end{solu}
\end{mathques}

\begin{mathques}{2020-09-28}{一元函数几何应用}[极坐标系下的曲线需要转化成直角
  坐标系下的参数方程后,再求切线方程]
\begin{ques}
  曲线$r = \cos 2\theta$在$\theta = \frac{\pi}{4}$处的切线方程为\mathblank.
\end{ques}
\begin{solu}
  \answer{$y = x$}
  曲线参数方程:$
  \begin{dcases}
    x = \matherror{r\cos \theta} = \cos 2\theta\cos \theta, \\
    y = \matherror{r\sin \theta} = \cos 2\theta\sin \theta, \\
  \end{dcases}
  \theta = \frac{\pi}{4}
  $对应$(x_0, y_0) = (0, 0),$
  \[
    \eval{\dv{y}{x}}_{\theta = \frac{\pi}{4}} = \eval{\frac{-2\sin 2\theta \sin
    \theta + \cos 2\theta \cos \theta}{-2 \sin 2\theta \cos \theta - \cos
    2\theta \sin \theta}}_{\theta = \frac{\pi}{4}} = 1,
  \]
  切线方程为$y = x.$
\end{solu}
\end{mathques}

\begin{mathques}{2020-09-28}{一元函数几何应用}[手残计算错误]
\begin{ques}
  函数$y = (x - 1)^2 (x - 2)^2 (-3 \le x \le 4)$的值域是\mathblank.
\end{ques}
\begin{solu}
  \answer{$[0, 400]$}
  \begin{align*}
    \dv{y}{x} &= 2 (x - 1) (x - 2)^2 + 2(x - 1)^2 (x - 2) \\
    &= 2(x - 1) (x - 2) (2x - 3).
  \end{align*}
  令$\dv{y}{x} = 0$,得驻点:$x = 1, 2, \frac{3}{2},$
  \begin{align*}
    y_{\max} &= \max{y(1), y(2), y\qty(\frac{3}{2}), y(-3), y(4)} \\
    &= \max{0, 0, \frac{1}{16}, 400, 36} = 400, \\
    y_{\min} &= \min{y(1), y(2), y\qty(\frac{3}{2}), y(-3), y(4)} \\
    &= \max{0, 0, \frac{1}{16}, 400, 36} = 0,
  \end{align*}
  值域是$[0, 400]$
\end{solu}
\end{mathques}

\begin{mathques}{2020-09-28}{一元函数几何应用}[并不是二阶倒数取不到$0$值就没有
  拐点,只要发生正负变化即可\par 二阶导数无意义点原函数不一定无意义, 拐点的
  $y$值通过原函数求出]
\begin{ques}
  曲线$y = x^{\frac{5}{3}} + 3x + 5$的拐点坐标为\mathblank.
\end{ques}
\begin{solu}
  \answer{$(0, 5)$}

  $y = x^{\frac{5}{3}} + 3x + 5$的定义域为$(-\infty, +\infty)$.
  \begin{align*}
    y' &= \frac{5}{3}x^{\frac{2}{3}} + 3 \\
    y'' &= \frac{10}{9}x^{-\frac{1}{3}}.
  \end{align*}
  当$x = 0$时,$y''$不存在;

  当$x < 0$时,$y'' < 0$;

  当$x > 0$时,$y'' > 0$;

  当$x = 0$时,$y = 5$,即拐点为$(0, 5)$.
\end{solu}
\end{mathques}

\begin{mathques}{2020-09-28}{一元函数几何应用}[问的是极小值点不是极小值]
\begin{ques}
  设$y = x^3 - 3x^2 - 9x + 2$,则$y$在$[-2, 2]$上的最小值点为\mathblank.
\end{ques}
\begin{solu}
  \answer{$x = 2$}

  $y = x^3 - 3x^2 - 9x + 2$的定义域为$(-\infty, +\infty)$.

  令$y' = 0$得$y$的两个驻点$x_1 = -1, x_2 = 3(舍掉).$又
  \[
    \eval{y}_{x = -1} = 7\qc \eval{y}_{x = -2} = 0 \qc \eval{y}_{x = 2} = -20,
  \]
  比较上述三个值可知$y$在$[-2, 2]$上的最小值点为$x = 2.$
\end{solu}
\end{mathques}

\begin{mathques}{2020-09-28}{一元函数几何应用}[$b$计算错误]
\begin{ques}
曲线$y = \frac{x^2}{x + 2}$的斜渐近线为\mathblank.
\end{ques}
\begin{solu}
  \answer{$y = x - 2$}

  由于
  \begin{align*}
    a &= \lim_{x \to \infty} \dv{y}{x} = \lim_{x \to \infty} \frac{x^2}{x(x + 2)} =
    1, \\
    b &= \lim_{x \to \infty} (y - ax) \\
    &= \lim_{x \to \infty} \qty(\frac{x^2}{x + 2} - x) \\
    &= \lim_{x \to \infty} \frac{x^2 - x^2 - 2x}{x + 2} = -2
  \end{align*}
  因此所求斜渐近线方程为$y = x - 2.$
\end{solu}
\end{mathques}

\begin{mathques}(2){2020-10-28}{一元函数几何应用}[不知道曲率公式]
\begin{ques}
  曲线$4x^2 + y^2 = 4$在点$(0, 2)$处的曲率为\mathblank.
\end{ques}
\begin{solu}
  \answer{2}

  $4x^2 + y^2 = 4$两边对$x$求导可得$8x + 2yy' = 0$,则有$y' = -\frac{4x}{y}$,
  \[
    y'' = -4\cdot \frac{y - xy'}{y^2} = \frac{-4\qty(y + \frac{4x^2}{y})}{y^2}
    = -\frac{16}{y^3},
  \]
  点$(0, 2)$在曲线上,则$\eval{y''}_{x = 0} = -2, \eval{y'}_{x = 0} = 0$

  由曲率公式$k = \frac{\abs{y''}}{\qty(1 + (y')^2)^{\frac{3}{2}}}$,可得$k =
  2.$
\end{solu}
\end{mathques}

\begin{mathques}{2020-09-28}{一元函数几何应用}[极限不会求]
\begin{ques}
曲线$y = \frac{2x^2 - 3}{5x^2 + 2\sin x}$的水平渐近线为\mathblank.
\end{ques}
\begin{solu}
  \answer{$y = \frac{2}{5}$}

  \[
  \lim_{x \to \infty} y = \lim_{x \to \infty} \frac{2 - \frac{3}{x^2}}{5 +
  \frac{2}{x^2} \sin x} = \frac{2}{5}
  \]
  因此$y = \frac{2}{5}$为曲线的水平渐近线.
\end{solu}
\end{mathques}

\begin{mathques}{2020-09-29}{一元微分应用}[无思路]
\begin{ques}
  证明恒等式:$\arcsin x + \arccos x = \frac{\pi}{2} (-1 \le x \le 1)$.
\end{ques}
\begin{solu}
  \matherror*{构造函数$f(x) = \arcsin x + \arccos x, \abs{x} \le 1$},则其导数为
  \[
    f'(x) = \frac{1}{\sqrt{1 - x^2}} - \frac{1}{\sqrt{1 - x^2}} = 0,
  \]
  从而$f(x)$在其定义域上恒为常数,即$f(x) \equiv C$,$C$是常数.

  令$x = 0$得
  \[
    C = f(x) = \arcsin 0 + \arccos 0 = \frac{\pi}{2},
  \]
  即$\arcsin x + \arccos x = \frac{\pi}{2} (-1 \le x \le 1).$
\end{solu}
\end{mathques}

\begin{mathques}{2020-09-29}{一元微分应用}[无思路]
\begin{ques}
  若$\frac{a_n}{n + 1} + \frac{a_{n - 1}}{n} + \dotsb + a_0 = 0$,证明方程
  $a_n x^n + a_{n-1} x^{n-1} + \dotsb + a_0 = 0$在$(0, 1)$内至少有一个实根.
\end{ques}
\begin{solu}
  \mathidea{零点存在问题,用罗尔定理}

  记
  \[
  f(x) = \frac{a_n x^{n + 1}}{n + 1} + \frac{a_{n-1} x^n}{n} + \dotsb + a_0 x
  \quad\mathhint[r]{\text{构造函数}}
  \]
  则
  \[
    f'(x) = a_n x^n + a_{n-1} x^{n-1} + \dotsb + a_0 = 0.\quad\mathhint[r]
    {\text{欲求函数}}
  \]
  $f(x)$在$[0, 1]$上连续,在$(0, 1)$内可导,又$f(0) = 0, f(1) = \frac{a_n}{n +
  1} + \frac{a_{n - 1}}{n} + \dotsb + a_0 = 0 $,由罗尔定理知,方程$f'(x) = 0$
  即$a_n x^n + a_{n-1} x^{n-1} + \dotsb + a_0 = 0$在$(0, 1)$内至少有一个实根.
\end{solu}
\end{mathques}

\begin{mathques}(2){2020-10-03}{一元微分应用}[无法证明零点存在]
\begin{ques}
  设在$[1, +\infty)$上,$f''(x) < 0, f(1) = 2, f'(1) = -3.$ 证明:$f(x) = 0$在
  $(1, +\infty)$内只有一个实根.
\end{ques}
\begin{solu}
  \mathidea{零点唯一性可通过证明零点存在和函数单调}

  在$[1, +\infty)$上,$f''(x) < 0, f'(x)$单调减少,$f'(x) < f'(1) = -3 < 0$,
  所以$f(x)$在$[1, +\infty)$上单调减少.

  由拉格朗日中值定理当$x > 1$时,存在$\xi \in (1, \xi)$使得
  \[
    f'(\xi) = \frac{f(x) - f(1)}{x - 1}
  \]
  即
  \begin{align*}
    f(x) &= f(1) + f'(\xi)(x - 1) \\
    &< 2 + f'(1)(x - 1) \\
    & = 5 - 3x
  \end{align*}
  取$x = \frac{5}{3}$,则$f\qty(\frac{5}{3}) < 0$,又$f(1) = 2 > 0$,由零点定理得$f(x) = 0$
  在$(1, \frac{5}{3})$内至少有一个实根.

  综上所述,$f(x) = 0$在$(1, +\infty)$内有且只有一个实根.
\end{solu}
\end{mathques}

\begin{mathques}{2020-10-02}{一元微分物理应用}[变化速度是对$t$的导数]
\begin{ques}
  质点$P$沿抛物线$x = y^2(y > 0)$移动,$P$的横坐标$x$的变化速度为5\si{cm/s}.
  当$x = 9$时,点$P$到原点$O$的距离变化速度为\mathblank.
\end{ques}
\begin{solu}
  \answer{$\frac{95}{6\sqrt{10}}$\si{cm/s}}

  \mathidea{已知$x = y^2(y > 0), \dv{x}{t} = 5$,求$\eval{\dv{s}{t}}_{x = 9}$.}

  $P$到原点$O$的距离$s = \sqrt{x ^2 + y^2}$,则
  \[
    \dv{s}{t} = \dv{t}\sqrt{x^2 + y^2} = \dv{x}\sqrt{x^2 + y^2} \cdot
    \matherror{\dv{x}{t}} = \frac{5(2x + 1)}{2\sqrt{x^2 + x}}.
  \]
  当$x = 9$时,$\eval{\dv{x}{t}}_{x = 9} = \frac{95}{6\sqrt{10}}$ (\si{cm/s}).
\end{solu}
\end{mathques}

\begin{mathques}(2){2020-11-01}{一元微分物理应用}[计算错误]
\begin{ques}
  球的半径以5\si{cm/s}的速度匀速增长,问球的半径为50\si{cm}时,求的表面积和体
  积的增长速度各是多少?
\end{ques}
\begin{solu}
  表面积为$S = 4\pi r^2$,则
  \[
    \dv{S}{t} = 8\pi r\dv{r}{t} = 8\pi \cdot 50 \cdot 5 = 2000\pi (\si{cm^2 /
    s}).
  \]
  体积为$V = \frac{4}{3}\pi r^3$,则
  \[
    \dv{V}{t} = 4\pi r^2\dv{r}{t} = 4\pi \cdot (50)^2 \cdot 5 = 50000\pi (cm^3
    / s).
  \]
\end{solu}
\end{mathques}
\begin{mathques}{2020-10-09}{一元积分计算}[没想到可以直接凑微分计算代入计算判
  断是否收敛]
\begin{ques}
  下列反常积分中收敛的是(\quad).
\begin{multichoice}[4]
  \task $\int_{e}^{+\infty} \frac{\ln x}{x} \dd x$
  \task $\int_{e}^{+\infty} \frac{\dd x}{x \ln x}$
  \task \answer{$\int_{e}^{+\infty} \frac{\dd x}{x (\ln x)^2}$}
  \task $\int_{e}^{+\infty} \frac{\dd x}{x(\ln x)^{\frac{1}{2}}}$
\end{multichoice}
\end{ques}
\begin{solu}
\begin{multichoice}[1]
  \task $\int_{e}^{+\infty} \frac{\ln x}{x} = \eval{\frac{1}{2} (\ln x)^2}_{e}
  ^{+\infty}$,发散
  \task $\int_{e}^{+\infty} \frac{\dd x}{x \ln x} = \int_{e}^{+\infty}
  \frac{\dd (\ln x)}{\ln x} = \eval{\ln(\ln x)}_{e}^{+\infty}$,发散
  \task $\int_{e}^{+\infty} \frac{\dd x}{x (\ln x)^2} = \int_{e}^{+\infty}
  \frac{\dd (\ln x)}{(\ln x)^2} = \eval{-\frac{1}{\ln x}}_{e}^{+\infty} = - (0
  - 1) = 1$,收敛
  \task $\int_{e}^{+\infty} \frac{\dd x}{x(\ln x)^{\frac{1}{2}}} = \int_{e}
  ^{+\infty} (\ln x)^{-\frac{1}{2}} \dd (\ln x) = \eval{2(\ln x)^{\frac{1}{2}}}
  _{e}^{+\infty}$,发散
\end{multichoice}
\end{solu}
\end{mathques}

\begin{mathques}(2){2020-10-13}{一元积分计算}[对于反三角函数分部积分不熟悉]
\begin{ques}
$\int \frac{\arcsin \sqrt{x}}{\sqrt{x}}\dd x =$\mathblank.
\end{ques}
\begin{solu}
  \mathidea{去掉根号会使计算变得简单}

  令$\sqrt{x} = t$,则$x = t^2$,则
  \begin{align*}
    \int \frac{\arcsin \sqrt{x}}{\sqrt{x}}\dd x &= 2 \int t \cdot
    \frac{\arcsin t}{t} \dd t\\
    &= 2 \int \arcsin t \dd t\\
    &= 2\matherror{(t\arcsin t + \sqrt{1 - t^2})} + C\\
    &= 2\sqrt{x}\arcsin \sqrt{x} + 2\sqrt{1 - x} + C
  \end{align*}
\end{solu}
\end{mathques}

\begin{mathques}(2){2020-10-13}{一元积分计算}[漏了$C$]
\begin{ques}
$\int e^{\sqrt{2x + 1}} \dd x = $\mathblank.
\end{ques}
\begin{solu}
  \answer{$(\sqrt{2x + 1} - 1)e^{\sqrt{2x + 1}} + C$}
  \mathmethod 令$\sqrt{2x + 1} = t$,则
  \begin{align*}
    \int e^{\sqrt{2x + 1}} &= \int te^t \dd t = \int t \dd (e^t)\\
    \mathidea{\small 分部积分法} &= te^t - e^t + C\\
    &= (\sqrt{2x + 1} - 1)e^{\sqrt{2x + 1}} + C
  \end{align*}
  \mathmethod 凑微分法
  \begin{align*}
    \int e^{\sqrt{2x + 1}} &= \int \sqrt{2x + 1}e^{\sqrt{2x + 1}} \dd
    (\sqrt{2x + 1})\\
    &= \int \sqrt{2x + 1} \dd (e ^{\sqrt{2x + 1}})\\
    \mathidea{\small 分部积分} &= \sqrt{2x + 1} e^{\sqrt{2x + 1}} - \int
    e^{\sqrt{2x + 1}} \dd (\sqrt{2x + 1})\\
    &= \sqrt{2x + 1}e^{\sqrt{2x + 1}} - e^{\sqrt{2x + 1}} + C\\
    &= (\sqrt{2x + 1} - 1)e^{\sqrt{2x + 1}} + C
  \end{align*}
\end{solu}
\end{mathques}

\begin{mathques}(5){2020-10-24}{一元积分计算}[无思路,换元法和分部积分不熟悉]
\begin{ques}
  设$x > 1$,则$\int \frac{xe^x}{\sqrt{e^x - 2}}\dd x = $\mathblank.
\end{ques}
\begin{solu}
  \answer{$2(x - 2)\sqrt{e^x - 2} + 4\sqrt{2}\arctan \sqrt{\frac{e^x}{2} - 1}
  + C$}

  令$\sqrt{e^2 - 2} = t$

  \begin{minipage}{\linewidth}
  \begin{align*}
    \int \frac{xe^x}{\sqrt{e^x - 2}}\dd x &= 2\int \ln(t^2 + 2) \dd t\\
    \mathidea{分部积分} &= \matherror{2t}\cdot\ln(t^2 + 2) -
    \mathmark[blue!30!cyan](t2-a){
      4\int \frac{t^2}{t^2 + 2} \dd t
    }\\
    &= 2t\ln(t^2 + 2) - 4t + 4\sqrt{2}\arctan \frac{t}{\sqrt{2}} + C\\
    &= 2(x - 2)\sqrt{e^x - 2} + 4\sqrt{2}\arctan \sqrt{\frac{e^x}{2} - 1} + C
  \end{align*}
  \[
  \mathmark[blue!30!cyan][
  \path[draw=blue!30!cyan, arrows={-Implies[]}, thick, opacity=0.5, double distance=2mm]
      (t2-a.east) to[bend left=90, distance=3cm] (frame.east);
  ]{
    \begin{aligned}
      \int \frac{t^2}{t^2 + 2} \dd t &= \int \qty(1 - \frac{1}{\qty(\frac{t}
      {\sqrt{2}})^2 + 1})\dd t\\
      &= t - \sqrt{2}\int \frac{\dd (\frac{t}{\sqrt{2}})}{\qty(\frac{t}
      {\sqrt{2}})^2 + 1}\\
      &= t - \sqrt{2}\arctan \frac{t}{\sqrt{2}} + C
    \end{aligned}
  }
  \]
  \end{minipage}
\end{solu}
\end{mathques}

\begin{mathques}(3){2020-10-15}{一元积分计算}[无思路,三角函数积分变换不熟悉]
\begin{ques}
$\int \frac{\dd x}{1 + \sin x + \cos x} =$\mathblank.
\end{ques}
\begin{solu}
  \answer{$\ln \abs{1 + \tan \frac{x}{2}} + C$}
  \begin{align*}
    \int \frac{\dd x}{1 + \sin x + \cos x} &= \int \frac{\dd x}{2\sin \frac{x}
      {2}\cos \frac{x}{2} + 2\cos[2](\frac{x}{2})}\\
      &= \frac{1}{2}\int \frac{\dd x}{\cos[2](\frac{x}{2})\cdot \qty(1 + \tan
      \frac{x}{2})}\\
      &= \int \frac{\dd (1 + \tan \frac{x}{2})}{1 + \tan \frac{x}{2}}\\
      &= \ln \abs{1 + \tan \frac{x}{2}} + C
  \end{align*}
\end{solu}
\end{mathques}

\begin{mathques}{2020-10-11}{一元积分计算}[三角函数变换不熟悉]
\begin{ques}
  求不定积分$\int \frac{x\cos[4](\frac{x}{2})}{\sin[3](x)}\dd x$
\end{ques}
\begin{solu}
  \begin{mathideabox}
  当被积函数为幂函数与三角函数的乘积,且不能用凑微分法积分时,一定要用分部积分
  法积分。\par
  因为分子与分母中三角函数的角度分别为$\frac{x}{2}$和$x$,所以首先要用三角恒等
  式变成同角度的三角函数形式。
  \end{mathideabox}
  \mathmethod
  \begin{align*}
    \int \frac{x\cos[4](\frac{x}{2})}{\sin[3](x)}\dd x &= \int \frac{x\cos[4]
    (\frac{x}{2})}{8\sin[3](\frac{x}{2})\cos[3](\frac{x}{2})}\dd x \\
    &= \int \frac{x\cos \frac{x}{2}}{8\sin[3](\frac{x}{2})}\dd x\\
    &= \matherror{\frac{1}{4}\int x \sin[-3](\frac{x}{2}) \dd(\sin \frac{x}{2})}
    \\
    &= \matherror{-\frac{1}{8}\int x\dd(\sin[-2](\frac{x}{2}))}\\
    &= \frac{-x}{8\sin[2](\frac{x}{2})} + \matherror{\frac{1}{8}\int \frac{\dd
    x}{\sin[2](\frac{x}{2})}}\\
    &= \frac{-x}{8\sin[2](\frac{x}{2})} - \matherror{\frac{1}{4}\cot \frac{x}
    {2}} + C
  \end{align*}
  \mathmethod
  \begin{align*}
    \int \frac{x\cos[4](\frac{x}{2})}{\sin[3](x)}\dd x &= \int \frac{x\cos
    \frac{x}{2}}{8\sin[3](\frac{x}{2})}\dd x\\
    &= -\frac{1}{4}\int x \cot \frac{x}{2} \dd(\cot \frac{x}{2})\\
    &= -\frac{1}{8}\int x\dd(\cot[2](\frac{x}{2}))\\
    &= -\frac{1}{8}x\cot[2](\frac{x}{2}) + \frac{1}{8}\int \cot[2](\frac{x}{2})
    \dd x \\
    &= -\frac{1}{8}x\cot[2](\frac{x}{2}) + \frac{1}{8} \int \qty(\csc[2]
    (\frac{x}{2}) - 1)\dd x\\
    &= \frac{-x}{8\sin[2](\frac{x}{2})} - \frac{1}{4}\cot \frac{x}{2} + C
  \end{align*}
\end{solu}
\end{mathques}

\begin{mathques}(5){2020-11-04}{一元积分计算}[没有想到把整个式子直接进行换元]
\begin{ques}
  计算$\int \arcsin \sqrt{\frac{x}{a + x}}\dd x (a > 0\text{是常数})$.
\end{ques}
\begin{solu}
  令$\arcsin \sqrt{\frac{x}{a + x}} = t, x = \frac{a\sin[2](t)}{1 - \sin[2](t)}
  = a\tan[2](t)$,则
  \begin{align*}
    \int \arcsin \sqrt{\frac{x}{a + x}}\dd x &= \int t\dd(a\tan[2](t))\\
    &= at\tan[2](t) - a\int \matherror{\tan[2](t)} \dd t\\
    &= at\tan[2](t) + a\int \matherror{\qty(1 - \sec[2](t))}\dd t\\
    &= at\tan[2](t) + at - a\tan t + C\\
    &= (a + x)\arcsin \sqrt{\frac{x}{a + x}} - \sqrt{ax} + C
  \end{align*}
\end{solu}
\end{mathques}

\begin{mathques}(2){2020-10-12}{一元积分计算}[不需要已知分部积分到底,只需要分到
  可以积分即可]
\begin{ques}
  求不定积分$\int \frac{\arctan e^x}{e^x} \dd x$
\end{ques}
\begin{solu}
  \mathidea{不满足凑微分法的条件,用分部积分}

  \mathmethod
  \begin{align*}
    \int \frac{\arctan e^x}{e^x} \dd x &= -\int \arctan e^x \dd(e^{-x})\\
    &= -e^{-x}\arctan e^x + \int \frac{e^{-x}\cdot e^x}{1 + e^{2x}} \dd x\\
    &= -e^{-x}\arctan e^x + \int \qty(1 - \frac{e^{2x}}{1 + e^{2x}}) \dd x\\
    &= -e^{-x}\arctan e^x + x - \frac{1}{2}\ln(1 + e^{2x}) + C
  \end{align*}
  \mathmethod 令$e^x = t$,则$x = \ln t, \dd x = \frac{1}{t} \dd t$
  \begin{align*}
    \int \frac{\arctan e^x}{e^x} \dd x &= \int \frac{\arctan t}{t^2} \dd t\\
    &= - \int \arctan t\dd(\frac{1}{t})\\
    &= - \frac{1}{t}\arctan t + \matherror{\int \frac{\dd t}{t(1 + t^2)}}\\
    &= -\frac{1}{t}\arctan t + \matherror{\int \frac{\dd t}{t} - \int
    \frac{t\dd t}{1 + t^2}}\\
    &= -\frac{1}{t}\arctan t + \ln t - \frac{1}{2}\ln(1 + t^2) + C\\
    &= -\frac{1}{e^x}\arctan e^x + x - \frac{1}{2} \ln(1 + e^{2x}) + C
  \end{align*}
\end{solu}
\end{mathques}

\begin{mathques}(6){2020-10-22}{一元积分计算}[没想到需要分类讨论,对三角函数积分
  不熟悉]
\begin{ques}
  计算不定积分$\int \frac{\tan x}{a^2\sin[2](x) + b^2\cos[2](x)}\dd x(a,
  b\text{不同时为零})$.
\end{ques}
\begin{solu}
\begin{solulist}
  \item 当$a = 0, b \neq 0$时,原式$= \frac{1}{b^2} \int \frac{\sin x}
    {\cos[3](x)}\dd x = \frac{1}{2b^2\cos[2](x)} + C$;
  \item 当$a \neq 0, b = 0$时,
    \begin{align*}
      \text{原式} &= \frac{1}{a^2}\int \frac{1}{\sin x\cos x}\dd x\\
      &= \frac{2}{a^2}\int \frac{1}{\sin 2x}\dd x\\
      &= \matherror{\frac{1}{a^2} \int \csc 2x \dd (2x)}\\
      &= \frac{1}{a^2}\ln\abs{\csc 2x - \cot 2x} + C
    \end{align*}
  \item 当$ab \neq 0$时,
    \begin{align*}
      \text{原式} &= \int \frac{\tan x}{\cos[2](x)\qty{a^2 \tan[2](x) + b^2}}
      \dd x\\
      &= \matherror{\int \frac{\tan x}{a^2\tan[2](x) + b^2}\dd(\tan x)}\\
      &= \matherror{\frac{1}{2a^2}\int \frac{\dd(a^2 \tan[2](x) + b^2)}{a^2
      \tan[2](x) + b^2}}\\
      &= \frac{1}{2a^2}\ln(a^2 \tan[2](x) + b^2) + C
    \end{align*}
\end{solulist}
\end{solu}
\end{mathques}

\begin{mathques}(2){2020-10-12}{一元积分计算}[夹逼准则和凑定积分可以一起使用]
\begin{ques}
求极限$\lim_{n \to \infty} \sum_{i = 1}^n \frac{\sin \frac{i\pi}{n}}{n +
\frac{1}{i}}$.
\end{ques}
\begin{solu}
  当各项分母相同且均为$n$时,
  \[
  \lim_{n \to \infty} \sum_{i = 1}^n \frac{\sin \frac{i\pi}{n}}{n + \frac{1}
  {i}} = \lim_{n \to \infty} \frac{1}{n}\sum_{i = 1}^n \sin \frac{i}{n}\pi =
  \int_{0}^{1} \sin \pi x\dd x.
  \]
  先对$\sum_{i = 1}^n \frac{\sin \frac{i\pi}{n}}{n + \frac{1}{i}}$进行放缩,有
  \[
    \sum_{i = 1}^n \frac{\sin \frac{i\pi}{n}}{n + 1} \le \sum_{i = 1}^n
    \frac{\sin \frac{i\pi}{n}}{n + \frac{1}{i}} \le \sum_{i = 1}^n \frac{\sin
    \frac{i\pi}{n}}{n},
  \]
  又
  \begin{align*}
    \lim_{n \to \infty} \sum_{i = 1}^n \frac{\sin \frac{i\pi}{n}}{n} &=
    \lim_{n \to \infty} \matherror{\frac{1}{n}} \sum_{i = 1}^n \matherror{\sin
    \frac{i} {n}\pi} = \int_{0}^{1}\sin \pi x\dd x = \frac{2}{\pi},\\
    \lim_{n \to \infty} \sum_{i = 1}^n \frac{\sin \frac{i\pi}{n}}{n + 1} &=
    \lim_{n \to \infty} \matherror{\frac{n}{n + 1} \cdot \frac{1}{n}} \sum_{i
    = 1}^n \sin \frac{i} {n}\pi = \int_{0}^{1}\sin \pi x\dd x =
    \frac{2}{\pi},
  \end{align*}
  由\matherror*{夹逼准则}即得
  \[
  \lim_{n \to \infty} \sum_{i = 1}^n \frac{\sin \frac{i\pi}{n}}{n + \frac{1}
  {i}} = \frac{2}{\pi}.
  \]
\end{solu}
\end{mathques}

\begin{mathques}(3){2020-11-11}{一元积分计算}[没能求出正确的通项公式来凑积分]
\begin{ques}
  $\lim_{n \to \infty} \qty(\frac{1}{n} \cdot \frac{2}{n} \cdot \dotsi \cdot
  \frac{n}{n})^{\frac{1}{n}} =$\mathblank.
\end{ques}
\begin{solu}
  \answer{$\frac{1}{e}$}

  令$x_n = \lim_{n \to \infty} \qty(\frac{1}{n} \cdot \frac{2}{n} \cdot \dotsi
  \cdot \frac{n}{n})^{\frac{1}{n}}$,则
  \[
    \ln x_n = \frac{1}{n}\qty(\ln \frac{1}{n} + \ln\frac{2}{n} + \dotsb + \ln
    \frac{n}{n}) = \sum_{k = 1}^n \frac{1}{n} \ln \frac{k}{n},
  \]
  于是
  \begin{align*}
    \lim_{n \to \infty} \ln x_n &= \lim_{n \to \infty} \sum_{k = 1}^n \frac{1}
    {n} \ln \frac{k}{n}\\
    &= \int_{0}^{1} \ln x\dd x\\
    &= \eval{x \ln x}_0^1 - \int_{0}^{1}x \cdot \frac{1}{x} \dd x\\
    &= 0 - 1 = -1.
  \end{align*}
  故$\lim_{n \to \infty} x_n = e^{-1} = \frac{1}{e}.$
\end{solu}
\end{mathques}

\begin{mathques}(4){2020-10-18}{一元积分计算}[没有想到正确的换元方法]
\begin{ques}
  计算定积分$\int_{\frac{1}{2}}^{\frac{3}{2}} \frac{(1 - x)\arcsin(1 - x)}
  {\sqrt{2x - x^2}}\dd x.$
\end{ques}
\begin{solu}
  令$1 - x = \sin t$,则
  \begin{align*}
  \int_{\frac{1}{2}}^{\frac{3}{2}} \frac{(1 - x)\arcsin(1 - x)} {\sqrt{2x -
  x^2}}\dd x &= \int_{-\frac{\pi}{6}}^{\frac{\pi}{6}} \frac{t\sin t}{\cos t}
  \cos t\dd t\\
  &= \int_{-\frac{\pi}{6}}^{\frac{\pi}{6}} t\sin t\dd t\\
  &= -\int_{-\frac{\pi}{6}}^{\frac{\pi}{6}}t\dd(\cos t)\\
  &= \eval{-(t\cos t - \sin t)}_{-\frac{\pi}{6}}^{\frac{\pi}{6}}\\
  &= 1 - \frac{\sqrt{3}\pi}{6}.
  \end{align*}
\end{solu}
\end{mathques}

\begin{mathques}(4){2020-11-18}{一元积分计算}[无思路]
\begin{ques}
计算定积分$\int_{-\frac{\pi}{4}}^{\frac{\pi}{4}} e^{\frac{x}{2}} \frac{\cos
x - \sin x}{\sqrt{\cos x}}\dd x.$
\end{ques}
\begin{solu}
  \begin{mathideabox}
  当被积函数为指函数与三角函数的乘积时,不能使用凑微分法求出原函数,只能使用分
  部积分法
  \end{mathideabox}
  \begin{align*}
  \int_{-\frac{\pi}{4}}^{\frac{\pi}{4}} e^{\frac{x}{2}} \frac{\cos x - \sin
  x}{\sqrt{\cos x}}\dd x &= \int_{-\frac{\pi}{4}}^{\frac{\pi}{4}} e^{\frac{x}
{2}} \sqrt{\cos x}\dd x - \int_{-\frac{\pi}{4}}^{\frac{\pi}{4}} e^{\frac{x}
{2}} \frac{\sin x}{\sqrt{\cos x}}\dd x\\
  &= \int_{-\frac{\pi}{4}}^{\frac{\pi}{4}} e^{\frac{x}{2}}\sqrt{\cos x}\dd x
  + 2\int_{-\frac{\pi}{4}}^{\frac{\pi}{4}} e^{\frac{x}{2}}\dd (\sqrt{\cos x})
  \\
  &= \int_{-\frac{\pi}{4}}^{\frac{\pi}{4}} e^{\frac{x}{2}}\sqrt{\cos x}\dd x
  + \eval{2e^{\frac{x}{2}}\sqrt{\cos x}}_{-\frac{\pi}{4}}^{\frac{\pi}{4}} -
  \matherror{\int_{-\frac{\pi}{4}}^{\frac{\pi}{4}}e^{\frac{x}{2}} \sqrt{\cos
  x} \dd x}\\
  &= \sqrt[4]{8}\qty(e^{\frac{\pi}{8}} - e^{-\frac{\pi}{8}}).
  \end{align*}
\end{solu}
\end{mathques}

\begin{mathques}(3){2020-10-14}{一元积分计算}[没有想到可以通过平移区间来化简]
\begin{ques}
  计算定积分$I = \int_{0}^{\pi}\frac{x\sin x}{1 + \cos[2](x)}\dd x.$
\end{ques}
\begin{solu}
  \begin{mathideabox}
  被积函数中既有幂函数,又有三角函数,是一种较复杂的分式形式,很难直接求出其原
  函数,但是,注意到被积函数为偶函数,其中$x$及$\sin x$为奇函数,可考虑将其化
  为对称区间上的定积分,为此,需将积分区间$[0, \pi]$平移成$\qty[-\frac{\pi}{2},
  \frac{\pi}{2}]$
  \end{mathideabox}
  设$x = \frac{\pi}{2} + u$,则有
  \begin{align*}
    I &= \int_{0}^{\pi}\frac{x\sin x}{1 + \cos[2](x)}\dd x\\
    &= \int_{-\frac{\pi}{2}}^{\frac{\pi}{2}} \frac{\qty(\frac{\pi}{2} + u)
    \cos u}{1 + \sin[2](u)}\dd u\\
    &= \frac{\pi}{2}\int_{-\frac{\pi}{2}}^{\frac{\pi}{2}}
    \mathmark[red][
    \node[red, font=\small] (a) at ([shift={(1.5, 1)}]frame.north east) {偶函数};
    \draw[->, red] (frame.north east) to[bend left] (a);
    ]{
      \frac{\cos u}{1 + \sin[2](u)}
    }
    \dd u + \int_{-\frac{\pi}{2}}^{\frac{\pi}{2}}
    \mathmark[cyan][
    \node[cyan, font=\small] (b) at ([shift={(1, 1)}]frame.north east) {奇函数};
    \draw[->, cyan] (frame.north) to[bend left] (b);
    ]{
      \frac{u\cos u}{1 + \sin[2](u)}
    }\dd u\\
  &= \pi\int_{0}^{\frac{\pi}{2}} \frac{\dd (\sin u)}{1 + \sin[2](u)} + 0\\
  &= \eval{\pi \arctan(\sin u)}_{0}^{\frac{\pi}{2}}\\
  &= \pi \times \frac{\pi}{4} = \frac{1}{4} \pi^2.
  \end{align*}
\end{solu}
\end{mathques}

\begin{mathques}(4){2020-10-16}{一元积分计算}[计算错误,三角函数积分不熟悉]
\begin{ques}
  设$f(x) =
\begin{dcases}
  \frac{1}{1 + \sin x}, & x \ge 0,\\
  \frac{1}{1 + e^x}, & x < 0,
\end{dcases}
$求$\int_{-1}^{\frac{\pi}{4}}f(x)\dd x.$
\end{ques}
\begin{solu}
  \[
    \int_{-1}^{\frac{\pi}{4}}f(x) \dd x = \int_{-1}^{0} f(x) \dd x + \int_{0}
    ^{\frac{\pi}{4}} f(x) \dd x = \int_{-1}^{0} \frac{\dd x}{1 + e^x} +
    \int_{0}^{\frac{\pi}{4}} \frac{\dd x}{1 + \sin x},
  \]
  令$e^x = t$,则
  \begin{align*}
    \int_{-1}^{0}\frac{\dd x}{1 + e^x} &= \int_{e^{-1}}^{1} \frac{1}{1 + t}
    \cdot \frac{1}{t} \dd t\\
    &= \int_{e^{-1}}^{1} \qty(\frac{1}{t} - \frac{1}{1 + t}) \dd t\\
    &= \eval{\ln \frac{t}{1 + t}}_{e^{-1}}^1\\
    &= -\ln 2 + \ln (1 + e),\\
    \int_{0}^{\frac{\pi}{4}}\frac{\dd x}{1 + \sin x} &= \int_{0}^{\frac{\pi}
    {4}} \frac{1 - \sin x}{\cos[2](x)} \dd x\\
    &= \int_{0}^{\frac{\pi}{4}} \sec[2](x)\dd x - \int_{0}^{\frac{\pi}{4}}
    \frac{\sin x}{\cos[2](x)} \dd x\\
    &= \eval{\tan x}_{0}^{\frac{\pi}{4}} - \eval{\frac{1}{\cos x}}_{0}
    ^{\frac{\pi}{4}} \\
    & = 1 - (\sqrt{2} - 1) = 2 - \sqrt{2}.
  \end{align*}
  从而
  \[
    \int_{-1}^{\frac{\pi}{4}} f(x)\dd x = - \ln 2 + \ln(1 + e) + 2 - \sqrt{2}.
  \]
\end{solu}
\end{mathques}

\begin{mathques}{2020-10-11}{一元积分计算}[当作复合函数求导,里面的也要求导]
\begin{ques}
  设下列不定积分存在,则下列命题正确的是(\quad).
  \begin{multichoice}
    \task \answer{$\int f'(2x)\dd x = \frac{1}{2}f(2x) + C$}
    \task $\qty[\int f(2x) \dd x]' = 2f(2x)$
    \task $\int f'(2x)\dd x = f(2x) + C$
    \task $\qty[\int f(2x) \dd x]' = \frac{1}{2}f(2x)$
  \end{multichoice}
\end{ques}
\begin{solu}
  令$t = 2x$,则$\dd x = \frac{1}{2}\dd t$,则
  \begin{align*}
    \int f'(2x)\dd x &= \frac{1}{2}\int f'(t) \dd t\\
    &= \frac{1}{2}f(t) + C\\
    &= \frac{1}{2}f(2x) + C
  \end{align*}
  故(A)选项正确
  \begin{align*}
    \qty[\int f(2x) \dd x]' &= \frac{\dd}{\matherror{\dd x}} \qty(\int f(2x)
    \dd x)\\
    &= \dv{t}(\frac{1}{2}\int f(t)\dd t)\cdot \dv{t}{x}\\
    &= \frac{1}{2}f(t) \cdot 2\\
    &= f(2x)
  \end{align*}
\end{solu}
\end{mathques}

\begin{mathques}{2020-10-11}{一元积分计算}[只要函数恒小于,积分也恒小于,就算
  是负的也是小于]
\begin{ques}
  设函数$f(x)$与$g(x)$在$[0, 1]$上连续,且$f(x) \le g(x)$,则对任意$c \in (0,
  1)$,有(\quad).
  \begin{multichoice}
    \task $\int_{\frac{1}{2}}^{c} f(t) \dd t \ge \int_{\frac{1}{2}}^{c} g(t)
    \dd t$
    \task $\int_{\frac{1}{2}}^{c} f(t) \dd t \le \int_{\frac{1}{2}}^{c} g(t)
    \dd t$
    \task $\int_{c}^{1} f(t) \dd t \ge \int_{c}^{1} g(t) \dd t$
    \task \answer{$\int_{c}^{1} f(t) \dd t \le \int_{c}^{1} g(t) \dd t$}
  \end{multichoice}
\end{ques}
\begin{solu}
  \begin{mathideabox}[定积分性质]
    若连续函数$f(x), g(x)$在$[a, b]$上满足$f(x) \le g(x)$,则当$a < b$时,
    $\int_{a}^{b}f(x) \dd x \le \int_{a}^{b}g(x)\dd x$
  \end{mathideabox}
  由于$c \in (0, 1)$,因此$c < 1$恒成立,而$c$可能大于$\frac{1}{2}$,也可能小
  于$\frac{1}{2}$,可知(A),(B)不正确。

  由于$f(x) \le g(x)$,可知应有$\int_{c}^{1}f(t) \dd t \le \int_{c}^{1} g(t)
  \dd t$.
\end{solu}
\end{mathques}

\begin{mathques}{2020-10-11}{一元积分计算}[$x$和$-x$图画错了orz]
\begin{ques}
  设$f(x) =
  \begin{dcases}
    1, & x > 0,\\
    0, & x = 0,\\
    -1, & x < 0,
  \end{dcases}
  F(x) = \int_{0}^{x}f(t)\dd t
  $,则(\quad).
  \begin{multichoice}[1]
    \task $F(x)$在$x = 0$处不连续
    \task \answer{$F(x)$在$(-\infty, +\infty)$内连续,但在$x = 0$处不可导}
    \task $F(x)$在$(-\infty, +\infty)$内可导,且满足$F'(x) = f(x)$
    \task $F(x)$在$(-\infty, +\infty)$内可导,但不一定满足$F'(x) = f(x)$
  \end{multichoice}
\end{ques}
\begin{solu}
  当$f(x)$为连续函数时,$F(x) = \int_{0}^{x}f(t)\dd t$为$f(x)$的原函数,此时有
  $\qty[\int_{0}^{x}f(t)\dd t]' = f(x)'$

  如果$f(x)$不为连续函数,上述结论不一定成立,因此不能轻易判定(C)成立。

  由于$f(x)$为分段函数,因此$F(x) = \int_{0}^{x}f(t) \dd t$也为分段函数

  当$x < 0$时,$F(x) = \int_{0}^{x}(-1)\dd t = -x$;

  当$x > 0$时,$F(x) = \int_{0}^{x}1\dd t$ = x;

  当$x = 0$时,$F(0) = 0$.

  因此$\matherror{F(x) = \abs{x}}$,可知$F(x)$在$(-\infty, +\infty)$内连续,但在
    $x = 0$处不可导。
\end{solu}
\end{mathques}

\begin{mathques}(3){2020-11-16}{一元积分计算}[凑定积分的公式理解错误]
\begin{ques}
  $\lim_{n \to \infty} \ln \sqrt[n]{\qty(1 + \frac{1}{n})^2 \qty(1 + \frac{2}
  {n})^2 \dotsi \qty(1 + \frac{n}{n})^2 } =$\mathblank.
\end{ques}
\begin{solu}
  \answer{$2(2\ln 2 - 1)$}
  \begin{align*}
    \text{原式} &= 2\lim_{n \to \infty} \frac{1}{n}\qty(\ln (1 + \frac{1}{n})
    + \ln(1 + \frac{2}{n}) + \dotsb + \ln(1 + \frac{n}{n}))\\
    &= 2\lim_{n \to \infty} \sum_{i = 1}^n \ln(1 + \frac{i}{n})\cdot \frac{1}
    {n}\\
    &= \matherror{2\int_{0}^{1} \ln(1+x)\dd x}\\
    &= \eval{2x\ln x}_1^2 - 2\int_{1}^{2} x \cdot \frac{1}{x} \dd x\\
    &= 2(2\ln 2 - 1).
  \end{align*}
\end{solu}
\end{mathques}

\begin{mathques}{2020-10-11}{一元积分计算}[凑定积分的公式理解错误]
\begin{ques}
  $\lim_{n \to \infty} \qty(\frac{1}{\sqrt{n^2 + n}} + \frac{1}{\sqrt{n^2 +
  2n}} + \frac{1}{\sqrt{n^2 + 3n}} + \dotsb + \frac{1}{\sqrt{n^2 + n^2}}) =
  $\mathblank.
\end{ques}
\begin{solu}
  \answer{$2\sqrt{2} - 2$}
  \begin{mathideabox}[凑定积分]
    根据定积分的定义凑定积分:$\int_{0}^{1}f(x)dx = \lim_{n \to \infty}
    \sum_{i = 1}^n f(\frac{i}{n}) \cdot \frac{1}{n}$

    能凑成$\frac{i}{n}$的用定积分定义;凑不成的,先用放缩法,常见的几种凑定积
    分定义的式子有
    \begin{flalign*}
      \lim_{n \to \infty} \qty(\frac{1}{n + 1} + \frac{1}{n + 2} + \dotsb
      + \frac{1}{n + n}) &= \lim_{n \to \infty} \sum_{i = 1}^n \frac{1}{n
      + i} &\\
      &= \lim_{n \to \infty} \sum_{i = 1}^n \frac{1}{1 + \frac{i}{n}}
      \cdot \frac{1}{n}\\
      &= \int_{0}^{1} \frac{1}{1 + x}\dd x\\
      &= \eval{\ln(1 + x)}_0^1 = \ln2.
    \end{flalign*}
    \begin{flalign*}
      \lim_{n \to \infty} \qty(\frac{n}{n^2 + 1^2} + \frac{n}{n^2 + 2^2} +
      \dotsb + \frac{n}{n^2 + n^2}) &= \lim_{n \to \infty} \sum_{i = 1}^n
      \frac{n}{n^2 + i^2} &\\
      &= \lim_{n \to \infty} \sum_{i = 1}^n \frac{n^2}{n^2 + i^2} \cdot
      \frac{1}{n}\\
      &= \lim_{n \to \infty} \sum_{i = 1}^n \frac{1}{1 + \qty(\frac{i}
      {n})^2} \cdot \frac{1}{n}\\
      &= \int_{0}^{1}\frac{1}{1 + x^2}\dd x\\
      &= \eval{\arctan x}_0^1 = \frac{\pi}{4}.
    \end{flalign*}
  \end{mathideabox}
  \begin{align*}
    \text{原式} &= \lim_{n \to \infty} \sum_{i = 1}^n \frac{1}{\sqrt{n^2 + ni}}
    \\
    &= \lim_{n \to \infty} \sum_{i = 1}^n \frac{1}{\sqrt{1 + \frac{i}{n}}}
    \cdot \frac{1}{n}\\
    &= \int_{0}^{1} \frac{1}{\sqrt{1 + x}}\dd x \\
    &= \eval{2\sqrt{1 + x}}_0^1 = 2\sqrt{2} - 2.
  \end{align*}
\end{solu}
\end{mathques}

\begin{mathques}{2020-10-14}{一元积分计算}[积分的方法不一样]
\begin{ques}
  不定积分$\int \frac{1}{\sqrt{x(4-x)}}\dd x =$\mathblank.
\end{ques}
\begin{solu}
  \answer{$\arcsin \frac{x - 2}{x} + C$}

  由于$\sqrt{x(4 - x)} = \sqrt{4 - (x - 2)^2} = 2\sqrt{1 - \qty(\frac{x - 2}
  {2})^2}$,所以
  \begin{align*}
    \int \frac{1}{\sqrt{x(4-x)}}\dd x &= \int \frac{\dd x}{2\sqrt{1 -
    \qty(\frac{x - 2}{2})^2}} \\
    &= \int \frac{\dd(\frac{x - 2}{2})}{\sqrt{1 - \qty(\frac{x - 2}{2})^2}}\\
    &= \arcsin \frac{x - 2}{2} + C.
  \end{align*}
\end{solu}
\end{mathques}

\begin{mathques}(2){2020-10-28}{一元积分计算}[没有想到可以把1变形]
\begin{ques}
  不定积分$\int \frac{1}{\sin[2](x)\cos[4](x)}\dd x =$\mathblank.
\end{ques}
\begin{solu}
  \answer{$\frac{1}{3}\tan[3](x) + 2\tan x - \cot x + C$}
  \begin{align*}
    \int \frac{1}{\sin[2](x)\cos[4](x)}\dd x &= \int \frac{\matherror{\qty(\sin[2]
    (x) + \cos[2](x))^2}}{\sin[2](x)\cos[4](x)} \dd x\\
    &= \int \frac{\sin[2](x)}{\cos[4](x)}\dd x + \int \frac{2}{\cos[2](x)}\dd
    x + \int \frac{1}{\sin[2](x)}\dd x\\
    &= \int \tan[2](x) \dd (\tan x) + 2\tan x - \cot x\\
    &= \frac{1}{3}\tan[3](x) + 2\tan x - \cot x + C.
  \end{align*}
\end{solu}
\end{mathques}

\begin{mathques}(3){2020-10-18}{一元积分计算}[抄错了orz]
\begin{ques}
  不定积分$\int \frac{\sin x - 3\cos x}{\sin[3](x)}\dd x =$\mathblank.
\end{ques}
\begin{solu}
  \answer{$-\cot x + \frac{3}{2\sin[2](x)} + C$}
  先将分式变形得
  \begin{align*}
    \int \frac{\sin x - 3\cos x}{\sin[3](x)}\dd x &= \int \frac{1}{\sin[2](x)}
    \dd x - 3\int \frac{\dd(\sin x)}{\sin[3](x)}\\
    &= -\cot x - 3\cdot \frac{1}{-3 + 1}\qty(\sin x)^{-3 + 1} + C\\
    &= -\cot x + \frac{3}{2\sin[2](x)} + C
  \end{align*}
\end{solu}
\end{mathques}

\begin{mathques}(3){2020-10-21}{一元积分计算}[思路错误]
\begin{ques}
不定积分$\int \frac{x^3}{\sqrt{1 - x^2}}\dd x =$\mathblank.
\end{ques}
\begin{solu}
  \answer{$\frac{1}{3}(1 - x^2)^{\frac{3}{2}} - (1 - x^2)^{\frac{1}{2}} + C$}

  \mathmethod 令$t = 1 - x^2$,则$\dd t = -2x\dd x$
  \begin{align*}
    \int \frac{x^3}{\sqrt{1 - x^2}}\dd x &= -\frac{1}{2}\int \frac{1 - t}
    {\sqrt{t}} \dd t\\
    &= -\frac{1}{2}\int \qty(\frac{1}{\sqrt{t}} - \sqrt{t}) \dd t\\
    &= \frac{1}{3}t^{\frac{3}{2}} - t^{\frac{1}{2}} + C\\
    &= \frac{1}{3}(1 - x^2)^{\frac{3}{2}} - (1 - x^2)^{\frac{1}{2}} + C
  \end{align*}

  \mathmethod 令$x = \sin t$,则
  \begin{align*}
    \int \frac{x^3}{\sqrt{1 - x^2}}\dd x &= \int \sin[3](t)\dd t\\
    &= \matherror{\int \sin[2](t)\cdot \sin t\dd t}\\
    &= \matherror{\int \qty(\cos[2](t) - 1) \dd (\cos t)}\\
    &= \frac{1}{3} \cos[3](t) - \cos t + C.
  \end{align*}
  利用$\cos t = \sqrt{1 - x^2}$得
  \[
  \text{原式}= \frac{1}{3}(1 - x^2)^{\frac{3}{2}} - (1 - x^2)^{\frac{1}{2}} + C
  \]
\end{solu}
\end{mathques}

\begin{mathques}{2020-10-15}{一元积分计算}[没有想到洛必达,以及分母也需要求导]
\begin{ques}
  $\lim_{x \to 0} \frac{\int_{0}^{x} \sin[2](t) \dd t}{x^3} =$\mathblank.
\end{ques}
\begin{solu}
  \answer{$\frac{1}{3}$}

  此极限属于“$\frac{0}{0}$”型,用洛必达法则,则
  \[
  \lim_{x \to 0} \frac{\int_{0}^{x} \sin[2](t) \dd t}{x^3} = \lim_{x \to 0}
  \frac{\sin[2](x)}{3x^2} = \frac{1}{3}.
  \]
\end{solu}
\end{mathques}

\begin{mathques}(2){2020-10-17}{一元积分计算}[中途积分计算错误,简单的可以直接求
  出没必要和复杂的一起变$\dd x$乘来乘去;倍角要化解成已知角进行还原]
\begin{ques}
求不定积分$\int x^2\sqrt{1 - x^2}\dd x$.
\end{ques}
\begin{solu}
  令$x = \sin t$,则
  \begin{align*}
    \int x^2\sqrt{1 - x^2}\dd x &= \int \sin[2](t)\cdot \cos t\dd(\sin t)\\
    &= \int \sin[2](t) \cdot \cos[2](t) \dd t\\
    &= \frac{1}{4}\int \sin[2](2t)\dd t\\
    &= \frac{1}{4}\int \frac{1}{2}(1 - \cos 4t)\dd t\\
    &= \frac{1}{8}t - \frac{1}{32}\sin 4t + C\\
    &= \frac{1}{8}\qty(\arcsin x - x\sqrt{1 - x^2}(1 - 2x^2)) + C.
  \end{align*}
\end{solu}
\end{mathques}

\begin{mathques}{2020-10-15}{一元积分计算}[没有对根号进行换元;反三角函数基本
  都需要进行分部积分后求出]
\begin{ques}
求不定积分$\int \frac{\arcsin \sqrt{x}}{\sqrt{x}}\dd x$.
\end{ques}
\begin{solu}
令$t = \sqrt{x}$,则$x = t^2$,即有
\begin{align*}
  \int \frac{\arcsin \sqrt{x}}{\sqrt{x}}\dd x &= 2\int \arcsin t\dd t\\
  &= 2\qty(t\cdot \arcsin t - \int \frac{t}{\sqrt{1 - t^2}}\dd t)\\
  &= 2\qty(t\cdot \arcsin t + \sqrt{1 - t^2}) + C\\
  &= 2\sqrt{x}\arcsin\sqrt{x} + 2\sqrt{1 - x} + C.
\end{align*}
\end{solu}
\end{mathques}

\begin{mathques}{2020-10-15}{一元积分计算}[计算过程中符号出错;像答案直接代入
  计算出错的可能性小些]
\begin{ques}
  设$xe^{-x}$为$f(x)$的一个原函数,求$\int xf(x)\dd x$.
\end{ques}
\begin{solu}
  因为$xe^{-x}$为$f(x)$的一个原函数,故
  \[
    f(x) = (xe^{-x})' = e^{-x}(1 - x).
  \]
  所以
  \begin{align*}
    \int xf(x)\dd x &= \int e^{-x}(x - x^2) \dd x\\
    \mathidea{分部积分}&= -e^{-x}(x - x^2) + \int e^{-x}(1 - 2x)\dd x\\
    \mathidea{再次分部积分}&= e^{-x}(x^2 - x) -e^{-x}(1 - 2x) - \int 2e^{-x}
    \dd x\\
    &= e^{-x}(x^2 + x + 1) + C.
  \end{align*}
\end{solu}
\end{mathques}

\begin{mathques}(3){2020-11-02}{一元积分计算}[凑反三角函数的公式不熟悉]
\begin{ques}
  求不定积分$\int \frac{\dd x}{(x^2 + 1)(x^2 + x + 1)}.$
\end{ques}
\begin{solu}
  \begin{mathideabox}[部分分式分解]
  设$f$和$g$是非零多项式,将$g$写作若干互不相同的不可约多项式的幂的乘积:
  \[
  g = \prod_{i = 1}^k p_i^{n_i}.
  \]
  则存在多项式$b$(不唯一)和多项式$a_{ij}$,且$\deg a_{ij} < \deg p_i$使得
  \[
  \frac{f}{g} = b + \sum_{i = 1}^k\sum_{j = 1}^{n_i} \frac{a_{ij}}{p_i^j}.
  \]
  如果$\deg f < \deg g$,则$b = 0$.
  \tcblower
  比如可设$\frac{3x + 5}{(1 - 2x)^2} = \frac{A}{(1 - 2x)^2} + \frac{B}{1 - 2x}
  .$
  则$3x + 5 = A + B(1 - 2x)$,解方程
  \[
  \begin{dcases}
  5 = A + B\\
  3x = -2Bx
  \end{dcases}
  \]
  得$A = \frac{13}{2}, B = -\frac{3}{2}$
  \end{mathideabox}
  由部分分式分解设
  \[
    \frac{1}{(x^2 + 1)(x^2 + x + 1)} = \frac{ax + b}{x^2 + 1} + \frac{cx + d}
    {x^2 + x + 1}
  \]
  消去分母得
  \[
    (a + c)x^3 + (a + b + d)x^2 + (a + b + c)x + (b + d) = 1
  \]
  解方程组
  \[
  \begin{dcases}
  a + c = 0,\\
  a + b + d = 0,\\
  a + b + c = 0,\\
  b + d = 1,
  \end{dcases}
  \]
  得$a = -1, b = 0, c=d = 1$
  则
  \begin{align*}
    \int \frac{\dd x}{(x^2 + 1)(x^2 + x + 1)} &= \int \qty(\frac{-x}{x^2 + 1}
    + \matherror{\frac{x + 1}{x^2 + x + 1}})\dd x\\
    &= -\frac{1}{2}\int \frac{\dd (x^2 + 1)}{x^2 + 1} + \matherror{\frac{1}
    {2}\int \frac{2x + 1 + 1}{x^2 + x + 1}\dd x} \\
    &= -\frac{1}{2}\ln(x^2 + 1) + \frac{1}{2}\int \frac{\dd (x ^2 + x + 1)}
    {x^2 + x + 1} + \frac{1}{2}\int \frac{\dd x}{x^2 + x + 1}\\
    &= -\frac{1}{2}\ln(x^2 + 1) + \frac{1}{2}\ln(x^2 + x + 1) + \frac{1}{2}
    \int \frac{\dd(x + \frac{1}{2})}{\qty(x + \frac{1}{2})^2 + \frac{3}{4}}\\
    &= -\frac{1}{2}\ln \frac{x^2 + 1}{x^2 + x + 1} +
    \matherror{\frac{\sqrt{3}}{3}\arctan \frac{2x + 1}{\sqrt{3}}} + C.
  \end{align*}
\end{solu}
\end{mathques}

\begin{mathques}{2020-10-15}{一元积分计算}[忘了有理函数的积分如何分解(见
  《30》p. 113)]
\begin{ques}
  求不定积分$\int \frac{1}{x^3 + 4x^2 + 5x + 2}\dd x.$
\end{ques}
\begin{solu}
  \begin{mathideabox}[试根法]
  对于复杂多项式,求其因式时,用试根法:

  对于整数多项式方程
  \[
  a_nx^n + a_{n - 1}x^{n - 1} + \dotsb + a_0 = 0
  \]
  设其有理数解为$x = \frac{p}{q}$,则
  $p$为$a_0$的因数,$q$为$a_n$的因数,且$p, q$互质(最大公约数为1)
  \tcblower
  对于$x^3 + 4x^2 + 5x + 2$,若有有理数根,则必定在$\pm1, \pm2$中,
  易得$-1$为其一个有理跟,则
  \[
    x^3 + 4x^2 + 5x + 2 = (x + 1)(x^2 + 3x + 2) = (x + 1)^2(x + 2)
  \]
  \end{mathideabox}
设$\frac{1}{x^3 + 4x^2 + 5x + 2} = \frac{1}{(x + 1)^2 (x + 2)} = \frac{A}{x +
2} + \frac{B}{x + 1} + \frac{C}{(x + 1)2}$

消去分母得
\[
  (A + B)x^2 + (2A + 3B +C)x + (A + 2B + 2C) = 1,
\]
解方程组
\[
A + B = 0,
2A + 3B + C = 0,
A + 2B + 2C = 1,
\]
得$A = 1, B = -1, C = 1$,则
\begin{align*}
  \int \frac{1}{x^3 + 4x^2 + 5x + 2}\dd x &= \int \frac{\dd x}{x + 2} - \int
  \frac{\dd x}{x + 1} + \int \frac{\dd x}{(x + 1)^2}\\
  &= \ln\abs{x + 2} - \ln\abs{x + 1} - \frac{1}{x + 1} + C.
\end{align*}
\end{solu}
\end{mathques}

\begin{mathques}{2020-10-18}{一元积分应用}[计算错误]
\begin{ques}
设曲线$y = x^2 - 2x, y = 0, x = 1, x = 3$围成一平面图形$A$,求:
\begin{queslist}
  \item $A$的面积
  \item 该平面图形绕$y$轴旋转一周所得旋转体的体积$V$.
\end{queslist}
\end{ques}
\begin{solu}
\columnratio{0.65}
\begin{paracol}{2}
\begin{rightcolumn}
\centering
\begin{tikzpicture}
\begin{axis}[
  math,
  width=\linewidth,
  xtick=\empty,
  ytick={-1, 3},
  xmax=4,
  ymin=-1.2
]
  \path[name path=xaxis] (\pgfkeysvalueof{/pgfplots/xmin}, 0) --
    (\pgfkeysvalueof{/pgfplots/xmax},0);
  \addplot[domain=-0.5:3.1, name path=A] {x^2 - 2*x};
  \addplot[gray, pattern=north east lines] fill between
    [of=A and xaxis, soft clip={domain=1:3}];
  \node[above left] at (1, 0) {$1$};
  \node[above left] at (2, 0) {$2$};
  \node[above right] at (3, 0) {$3$};
  \node[below left] at (0, 0) {$O$};
  \node[pin=-45:{$S_1$}, inner sep=0pt] at (1.4, -0.5) {};
  \node[pin=30:{$S_2$}, inner sep=0pt] at (2.8, 1.2) {};
  \draw (1, -1.5) -- (1, 3);
  \draw (3, -1.5) -- (3, 3);
  \draw[dashed] (0, 3) -- (3, 3);
\end{axis}
\end{tikzpicture}
\end{rightcolumn}
\begin{leftcolumn}
\begin{solulist}
  \item 画出图形,可见$S = S_1 + S_2$,其中
    \begin{align*}
      S_1 &= \int_{1}^{2} (2x - x^2)\dd x = \frac{2}{3},\\
      S_2 &= \int_{2}^{3} (x^2 - 2x)\dd x = \frac{4}{3},
    \end{align*}
    故$S = 2$.
  \item \mathmethod $V = V_1 + V_2$,其中
    \begin{align*}
      V_1 &= \pi\int_{-1}^{0}(1 + \sqrt{1 + y})^2 \dd y - \pi = \frac{11\pi}
      {6},\\
      V_2 &= 27\pi - \pi\int_{0}^{3}(1 + \sqrt{1 + y})^2 \dd y = \frac{43\pi}
      {6},
    \end{align*}
    故$V = 9\pi$
    \mathmethod
    \begin{align*}
      V_1 &= 2\pi\int_{1}^{2}x\qty(0 - (x^2 - 2x))\dd x = \frac{11\pi}{6},\\
      V_2 &= 2\pi\int_{2}^{3}x(x^2 - 2x)\dd x = \frac{43\pi}{6},
    \end{align*}
    故$V = 9\pi$
\end{solulist}
\end{leftcolumn}
\end{paracol}
\end{solu}
\end{mathques}

\begin{mathques}(3){2020-10-22}{一元积分应用}[$a$的正负大小并不确定,注意$y$的设
  法]
\begin{ques}
  已知一抛物线经过$x$轴上两点$A(1, 0), B(3, 0)$.
  \begin{queslist}
    \item 求证两坐标轴与该抛物线所围面积等于$x$轴与该抛物线所围图形的面积;
    \item 计算(1)中两个平面图形绕$x$轴旋转一周所产生的两个旋转体的体
      积之比.
  \end{queslist}
\end{ques}
\begin{solu}
\begin{solulist}
  \item \mathproof \mathmethod
    设过$A(1, 0), B(3, 0)$两点的抛物线方程为
    \[
      y=\matherror{a}(x - 1)(x - 3), \mathidea[r]{$a$正负号未定}
    \]
    则两坐标轴与该抛物线所围图形的面积为
    \[
      S_1 = \int_{0}^{1} \abs{a(x - 1)(x - 3)}\dd x = \abs{a}\int_{0}^{1}(x^2
      - 4x + 3)\dd x = \frac{4}{3}\abs{a}.
    \]
    $x$轴与该抛物线所围图形的面积为
    \[
      S_1 = \int_{1}^{3} \abs{a(x - 1)(x - 3)}\dd x = -\abs{a}\int_{1}^{3}(x^2
      - 4x + 3)\dd x = \frac{4}{3}\abs{a}.
    \]
    \mathmethod 因为$\int_{0}^{3} a(x - 1)(x-3)\dd x = \eval{a\qty(\frac{1}{3}
    x^3 - 2x^2 + 3x)}_{0}^3 = 0$,所以$S_1 = S_2$.
  \item 两个平面图形绕$x$轴旋转一周所产生的两个旋转体的体积分别为
    \begin{align*}
      V_1 &= \pi\int_{0}^{1}a^2\qty((x - 1)(x - 3))^2\dd x = \frac{38}{15}\pi
      a^2,\\
      V_2 &= \pi\int_{1}^{3}a^2\qty((x - 1)(x - 3))^2\dd x = \frac{16}{15}\pi
      a^2,
    \end{align*}
    故$\frac{V_1}{V_2} = \frac{19}{8}.$
\end{solulist}
\end{solu}
\end{mathques}

\begin{mathques}{2020-10-18}{一元积分应用}[算到后面把$\pi$漏了orz]
\begin{ques}
  求曲线$y = \sqrt{x}e^{-x}$与$x$轴所围成的平面图形在$[0, +\infty)$内肉$x$轴旋
  转一周所成的旋转体的体积.
\end{ques}
\begin{solu}
  由题意得,该平面图形绕$x$轴旋转一周所成的旋转体体积为
  \begin{align*}
    V_x &= \pi \int_{0}^{+\infty}\qty(\sqrt{x}e^{-x})^2\dd x
    = \pi\int_{0}^{+\infty}xe^{-2x}\dd x\\
    &= -\frac{1}{2}\pi\int_{0}^{+\infty}x \dd(e^{-2x})
    = -\frac{\pi}{2}\qty(\eval{xe^{-2x}}_0^{+\infty} - \int_{0}^{+\infty}
    e^{-2x}\dd x)\\
    &= -\frac{\pi}{2}\qty(0 + \eval{\frac{1}{2}e^{-2x}}_0^{+\infty}) =
    \frac{\pi}{4}.
  \end{align*}
\end{solu}
\end{mathques}

\begin{mathques}(3){2020-10-24}{一元积分应用}[没有考虑$a$小于$0$的结果;绕$x$轴旋
  转的体积是$y^2$之间的差不是$y$相减后的平方]
\begin{ques}
  设直线$y = ax$与抛物线$y = x^2$所围成的图形的面积为$S_1$,它们与直线$x = 1$
  所围成图形的面积为$S_2$,并且$a < 1$.
  \begin{queslist}
    \item 试确定$a$的值,使$S_1 + S_2$达到最小,并求出最小值;
    \item 求该最小值所对应的平面图形绕$x$轴旋转一周所得旋转体的体积.
  \end{queslist}
\end{ques}
\begin{solu}
  因为$a < 1$,所以可分成$\matherror{0 < a < 1, a \le 0}$两种情况,分别画出两
  种情况下的图形(见图 \ref{fig:int-10-18}),求出$S_1 + S_2$的最小值后,即可确
  定$a$的值.
  \begin{figure}[ht]
    \centering
    \subcaptionbox{$0 < a < 1$}[0.4\linewidth]{
\begin{tikzpicture}
\begin{axis}[
  math,
  width=\linewidth,
  xtick={1},
  ytick=\empty,
  extra x ticks={2 / 3},
  extra x tick labels={$a$},
  xticklabel style={text height=1.5ex},
  ymin=-0.7,
  xmax=1.5,
  ]
  \addplot[domain=0:1.1, name path=A] {x^2}
    node [pos=1, left, font=\small] {$y = x^2$}
    ;
  \addplot[domain=0:1.2, name path=B] {2/3*x}
    node [pos=0.9, below right, font=\small] {$y = ax$}
    ;
  \addplot fill between [
    of = A and B,
    soft clip={domain=0:1},
    split,
    every segment no 0/.style={fill=green!10},
    every segment no 1/.style={fill=red!10},
  ];

  \node[pin={[pin distance=3mm]120:{$S_1$}}, inner sep=0pt] at (3/8, 0.2) {};
  \node[pin={[pin distance=3mm]40:{$S_2$}}, inner sep=0pt] at (0.95, 0.8) {};
  \draw[dashed] (2/3, 0) -- (2/3, 4/9);
  \draw[dashed] (1, 0) -- (1, 2/3);
  \draw (1, 2/3) -- (1, 1);
\end{axis}
\end{tikzpicture}
    }
    \subcaptionbox{$a \le 1$}[0.4\linewidth]{
\begin{tikzpicture}
\begin{axis}[
  math,
  width=\linewidth,
  xtick=\empty,
  ytick=\empty,
  ymin=-0.7,
  ]
  \addplot[domain=-0.9:1.1, name path=A] {x^2}
    node [pos=1, left, font=\small] {$y = x^2$}
    ;
  \addplot[domain=-1:1.1, name path=B] {-0.6*x}
    node [pos=0.85, below left, font=\small] {$y = ax$}
    ;
  \addplot fill between [
    of = A and B,
    soft clip={domain=-0.6:1},
    split,
    every segment no 0/.style={fill=green!10},
    every segment no 1/.style={fill=red!10},
  ];

  \node[pin={[pin distance=3mm]90:{$S_1$}}, inner sep=0pt] at (-0.3, 0.12) {};
  \node[font=\small] at (0.8, 0.3) {$S_2$};
  \node[text height=2ex, below right] at (1, 0) {$1$};
  \node[text height=2ex, below] at (-0.6, 0) {$a$};
  \draw (1, -0.6) -- (1, 1);
  \draw[dashed] (-0.6, 0) -- (-0.6, 0.36);
\end{axis}
\end{tikzpicture}
    }
    \caption{示意图}
    \label{fig:int-10-18}
  \end{figure}
\begin{solulist}
  \item 当$0 < a < 1$时,
    \[
      S = S_1 + S_2 = \int_{0}^{a}(ax - x^2)\dd x + \int_{a}^{1}(x^2 - ax)\dd
      x = \frac{a^3}{3} - \frac{a}{2} + \frac{1}{3}.
    \]
    令$S'(a) = a^2 - \frac{1}{2} = 0$,求得$a = \frac{1}{\sqrt{2}}$. 又
    $S''\qty(\frac{1}{\sqrt{2}}) = \frac{2 - \sqrt{2}}{6}$是极小值,即最小值;

    当$\matherror{a \le 0}$时,
    \[
      S = S_1 + S_2 = \int_{a}^{0}(ax - x^2)\dd x + \int_{0}^{1}(x^2 - ax)\dd
      x = -\frac{a^3}{6} - \frac{a}{2} + \frac{1}{3}.
    \]
    因为
    \[
      S'(a) = -\frac{1}{2}a^2 - \frac{1}{2} < 0,
    \]
    所以$S$单调递减,故$a = 0$时,$S$取得最小值,此时$S = \frac{1}{3}$.

    比较可知,当$a = \frac{1}{\sqrt{2}}$时,$S\qty(\frac{1}{\sqrt{2}}) =
    \frac{2 - \sqrt{2}}{6}$是最小值.
  \item 由(1)可知旋转体体积为
    \[
      V_x = \pi\int_{0}^{\frac{1}{\sqrt{2}}} \qty(\frac{1}{2}x^2 - x^4) \dd x +
      \pi\int_{\frac{1}{\sqrt{2}}}^{1} \qty(x^4 - \frac{1}{2}x^2)\dd x =
      \frac{\sqrt{2} + 1}{30}\pi
    \]
\end{solulist}
\end{solu}
\end{mathques}

\begin{mathques}(2){2020-10-21}{一元积分应用}[没能正确将直角坐标系下的方程转化为
  极坐标系下的方程;对称性的判断不熟悉]
\begin{ques}
  双纽线$\qty(x^2 + y^2)^2 = x^2 - y^2$所围成的区域面积用定积分表示为(\quad).
  \begin{multichoice}
    \task \answer{$2\int_{0}^{\frac{\pi}{4}}\cos 2\theta\dd \theta$}
    \task $4\int_{0}^{\frac{\pi}{4}}\cos 2\theta\dd \theta$
    \task $2\int_{0}^{\frac{\pi}{4}}\sqrt{\cos 2\theta}\dd \theta$
    \task $\frac{1}{2}\int_{0}^{\frac{\pi}{4}}(\cos 2\theta)^2\dd \theta$
  \end{multichoice}
\end{ques}
\begin{solu}
  \begin{mathideabox}[对称性]
    对于$f(x, y) = 0$,若
    \begin{solulist}
      \item $f(x, -y) = 0$,则图像关于$x$轴对称
      \item $f(-x, y) = 0$,则图像关于$y$轴对称
      \item $f(-x, -y) = 0$,则图像关于原点对称
    \end{solulist}
    \tcbsubtitle[before skip=\baselineskip]{直角坐标转极坐标}
    \[
    \begin{dcases}
    x = r\cos\theta\\
    y = r\sin\theta
    \end{dcases}
    \]
  \end{mathideabox}
  双纽线的极坐标方程为$r^2 = \cos 2\theta$,根据对称性,所求面积为
  \[
  S = 4\cdot \frac{1}{2}\int_{0}^{\frac{\pi}{4}} r^2\dd \theta = 2\int_{0}
  ^{\frac{\pi}{4}}\cos 2\theta\dd \theta
  \]
\end{solu}
\end{mathques}

\begin{mathques}{2020-10-19}{一元积分应用}[算到后面忘了负号]
\begin{ques}
  函数$y = \frac{x}{\sqrt{1 - x^2}}$在区间$\qty[0, \frac{1}{2}]$上的平均值为
  \mathblank.
\end{ques}
\begin{solu}
  \answer{$2 - \sqrt{3}$}

  由平均值$\bar{y} = \frac{1}{b - a}\int_{a}^{b}y(x)\dd x$可知
  \begin{align*}
    \bar{y} &= \frac{1}{\frac{1}{2}}\int_{0}^{\frac{1}{2}} \frac{x}{\sqrt{1 -
    x^2}}\dd x\\
    &= - \int_{0}^{\frac{1}{2}}(1 - x^2)^{-\frac{1}{2}} \dd(1 - x^2)\\
    &=\eval{-2(1 - x^2)^{\frac{1}{2}}}_0^{\frac{1}{2}} = 2 - \sqrt{3}.
  \end{align*}
\end{solu}
\end{mathques}

\begin{mathques}{2020-10-19}{一元积分应用}[计算错误]
\begin{ques}
  设$y = x^2$定义在$[0, 2]$上,$t$为$[0, 2]$上任意一点,问当$t$为何值时,能使
  下图中阴影面积之和最小.
\par\hfill
\begin{tikzpicture}
\begin{axis}[
  math,
  xtick={2},
  ytick=\empty,
  width=0.3\linewidth,
  ymin=-0.5,
  xmin=-0.3
]
  \addplot[domain=0:2, name path=A] {x^2};
  \draw[name path=B] (0, 1.44) -- (2, 1.44);
  \addplot fill between [
    of = A and B,
    soft clip={domain=0:2},
    split,
    every segment no 0/.style={fill=green!10},
    every segment no 1/.style={fill=red!10},
  ];
  \draw[dashed] (1.2, 0) -- (1.2, 1.44);
  \draw (2, 0) -- (2, 4);
  \node[text height=1.5ex, below] at (1.2, 0) {$t$};
  \node[left] at (0, 1.44) {$t^2$};
  \node[font=\small] at (0.5, 1) {$S_1$};
  \node[font=\small] at (1.7, 2) {$S_2$};
\end{axis}
\end{tikzpicture}
\end{ques}
\begin{solu}
  \begin{align*}
    S_1 &= \int_{0}^{t}(t^2 - x^2)\dd x = \frac{2}{3}t^2,\\
    S_2 &= \int_{t}^{2}(x^2 - t^2)\dd x = \matherror{\frac{2}{3}t^3 - 2t^2 +
    \frac{8}{3}},\\
    S &= S_1 + S_2 = \frac{4}{3}t^3 - 2t^2 + \frac{8}{3},\\
    S' &= 4t^2 - 4t
  \end{align*}
  令$S' = 0$得$t = 0$或$t = 1$,
  \[
    S'' = 8t - 4\qc \matherror{\eval{S''}_{t = 0} = -4}\qc
    \matherror{\eval{S''}_{t = 1} = 4}.
  \]
  可知当$t = 1$时,$S$取极小值,\matherror*{由于在$(0, 2)$内驻点唯一,$S(1) =
  2$也是最小值.}
\end{solu}
\end{mathques}

\begin{mathques}{2020-10-19}{一元积分应用}[是外面的体积减去里面的体积,不是减
  了以后绕圈]
\begin{ques}
  设直线$y = ax(0 < a < 1)$与抛物线$y = x^2$所围成图形为$S_1$,它们相交后的部
  分与直线$x = 1$所围成图形为$S_2$. 确定$a$的值,使平面图形$S_1$与$S_2$绕$x$轴
  旋转一周所得旋转体体积之和最小.
\end{ques}
\begin{solu}
  当$0 < a < 1$时,直线$y = ax$倾角取值范围为$\qty(0, \frac{\pi}{4})$,它与抛
  物线下第一象限相交.

  由$
  \begin{dcases}
  y = x^2,\\
  y = ax,
  \end{dcases}
  $解得交点$(0, 0), (a, a^2)$.故旋转体体积之和为
  \[
    V = \int_{0}^{a} \pi(a^2x^2 - x^4)\dd x + \int_{a}^{1}\pi(x^4 - a^2x^2)\dd
    x = \pi\qty(\frac{4}{15}a^5 - \frac{1}{3}a^2 + \frac{1}{5}).
  \]
  又$V' = \frac{2}{3}\pi a(2a^3 - 1)$,令$V' = 0$得$a = \sqrt[3]{\frac{1}{2}}$,
  $a = 0$(舍),$V'' = \frac{2}{3}\pi(8a^3 - 1)$,$\eval{V''}_{a = \sqrt[3]
  {\frac{1}{2}}} > 0.$

  故$a = \sqrt[3]{\frac{1}{2}}$时,$V$取得极小值,由于驻点唯一,故$a = \sqrt[3]
  {\frac{1}{2}}$时,$V$取得最小值.
\end{solu}
\end{mathques}

\begin{mathques}(2){2020-10-21}{一元积分应用}[是外面的体积减去里面的体积,不是减
  了以后绕圈]
\begin{ques}
  过曲线$y = x^2$上点$(1, 1)$做切线,求切线方程,并求切线与曲线及$x$轴所围图形
  绕$x$轴旋转一周所得旋转体体积.
\end{ques}
\begin{solu}
  $\eval{y'}_{x = 1} = 2$,故过点$(1, 1)$切线方程为$y = 2x - 1$.

  该切线与$x$轴的交点为$\qty(\frac{1}{2}, 0)$,则旋转体体积为
  \[
    V = \pi\int_{0}^{1}x^4\dd x - \pi\int_{\frac{1}{2}}^{1}(2x - 1)^2\dd x =
    \frac{1}{5}\pi - \frac{1}{6}\pi = \frac{1}{30}\pi.
  \]
\end{solu}
\end{mathques}

\begin{mathques}{2020-10-23}{积分等式与不等式}[没有想到用$G(x)$的原函数;罗尔
  定理的方法]
\begin{ques}
  设$f(x), g(x)$在$[a, b]$上连续,证明至少存在一点$\xi \in (a, b)$,使得
  \[
    f(\xi)\int_{\xi}^{b}g(x)\dd x = g(\xi)\int_{a}^{\xi}f(x) \dd x
  \]
\end{ques}
\begin{solu}
  \begin{mathideabox}
    若令$G(x) = f(x)\int_{x}^{b}g(t)\dd t - g(x) \int_{a}^{x}f(t)\dd t$,无法
    验证其满足零点定理,所以应求$G(x)$的原函数,使用罗尔定理证明。
  \end{mathideabox}
  \mathproof 设$F(x)$为$G(x) = f(x)\int_{x}^{b}g(t)\dd t - g(x) \int_{a}^{x}
  f(t)\dd t$的原函数,则
  \begin{align*}
    F(x) &= \int \qty(f(x)\int_{x}^{b}g(t)\dd t)\dd x - \int\qty(g(x)\int_{a}
    ^{x}f(t)\dd t)\dd x\\
    &= \int\qty(\int_{x}^{b}g(t)\dd t)\dd(\int_{a}^{x}f(t)\dd t) +
    \int\qty(\int_{a}^{x}f(t)\dd t)\dd(\int_{x}^{a}g(t)\dd t)\\
    &= \int_{a}^{x}f(t)\dd t\int_{x}^{b}g(t)\dd t
  \end{align*}
  因为$f(x), g(x)$在$[a, b]$上连续,所以$F(x)$在$[a, b]$上连续,在$(a, b)$内可
  导

  又$\matherror{F(a) = F(b) = 0}$,由罗尔定理可得,至少存在一点$\xi \in (a, b)
  $,使得$F'(\xi) = 0$,即
  \[
    f(\xi)\int_{\xi}^{b}g(x)\dd x = g(\xi)\int_{a}^{\xi}f(x) \dd x
  \]
\end{solu}
\end{mathques}

\begin{mathques}(2){2020-10-25}{积分等式与不等式}[卡在将非积分转化为变限积分的地
  方]
\begin{ques}
  设$f(x)$在$[a, b]$上连续且严格单调增加,证明
  \[
    (a + b)\int_{a}^{b}f(x)\dd x < 2\int_{a}^{b}xf(x)\dd x.
  \]
\end{ques}
\begin{solu}
  \mathproof 令$F(t) = (a + t)\int_{a}^{t}f(x)\dd x - 2\int_{a}^{t}xf(x)\dd x,
  t \in [a, b]$,则
  \begin{align*}
    F'(t) &= \int_{a}^{t}f(x)\dd x + (a + t)f(t) - 2tf(t)\\
    &= \int_{a}^{t}f(x)\dd x - \matherror{(t - a)f(t)}\\
    &= \int_{a}^{t}f(x)\dd x - \matherror{\int_{a}^{t}f(t)\dd x}\\
    &= \int_{a}^{t}\qty(f(x) - f(t))\dd x.
  \end{align*}
  因为$f(x)$在$[a, b]$上严格单调增加,所以$f(x) - f(t) < 0$,则有
  \[
    F'(t) = \int_{a}^{t}\qty(f(x) - f(t))\dd x < 0,
  \]
  即$F(t)$严格单调减少,又$F(a) = 0$,故$F(b) < 0$,即
  \[
    (a + b)\int_{a}^{b}f(x) \dd x < 2\int_{a}^{b}x f(x)\dd x.
  \]
\end{solu}
\end{mathques}

\begin{mathques}(2){2020-10-25}{积分等式与不等式}[卡在积分中值定理没有成功转换]
\begin{ques}
  设$f(x)$在$[0, 1]$上连续且单调递减,证明:当$0 < \lambda < 1$时,$\int_{0}
  ^{\lambda}f(x)\dd x \ge \lambda \int_{0}^{1}f(x)\dd x.$
\end{ques}
\begin{solu}
  要证原不等式成立,只需证$\frac{\int_{0}^{\lambda}f(x)\dd x}{\lambda} \ge
  \int_{0}^{1}f(x)\dd x$.

  令$F(t) = \frac{\int_{0}^{t}f(x)\dd x}{t}$,由于$F(\lambda) = \frac{\int_{0}
  ^{\lambda}f(x)\dd x}{\lambda}, F(1) = \int_{0}^{1}f(x)\dd x$,故只需证当
  $\lambda \in (0, 1)$时,有
  \begin{equation}
    F(\lambda) \ge F(1) \tag{$*$}
  \end{equation}
  $F(t)$在$(0, 1]$内连续,在$(0, 1)$内可导,且
  \[
    F'(t) = \frac{f(t)t - \int_{0}^{t}f(x)\dd x}{t^2} = \frac{f(t)t - f(c)t}
    {t^2} = \frac{f(t) - f(c)}{t},
  \]
  其中$0 \le c \le t$,则$f(c) \ge f(t)$,即有$F'(t) \le 0$,故$F(t)$在$(0, 1]
  $内单调递减.

  又$0 < \lambda < 1$,则$F(\lambda) \ge F(1)$,即$(*)$式成立
\end{solu}
\end{mathques}

\begin{mathques}(2){2020-10-25}{积分等式与不等式}[没有想到三角变换]
\begin{ques}
  设$f(x)$为$[0, 1]$上的连续函数,证明:
  \begin{queslist}
    \item $\int_{0}^{\frac{\pi}{2}}f(\sin x)\dd x = \int_{0}^{\frac{\pi}{2}}
      f(\cos x)\dd x$;
    \item $\int_{0}^{\pi}xf(\sin x)\dd x = \frac{\pi}{2}\int_{0}^{\pi}f(\sin x)
      \dd x$.
  \end{queslist}
  并计算$\int_{0}^{\pi}\frac{x\sin x}{1 + \cos[2](x)}\dd x$.
\end{ques}
\begin{solu}
  \begin{solulist}
    \item 设$\matherror{x = \frac{\pi}{2} - t}$,则$\dd x = -\dd t$,当$x = 0$
      时,$t = \frac{\pi}{2}$;当$x = \frac{\pi}{2}$时,$t = 0$. 则
      \[
        \int_{0}^{\frac{\pi}{2}}f(\sin x)\dd x = -\int_{\frac{\pi}{2}}^{0}
        f\qty(\sin(\frac{\pi}{2} - t))\dd t = \int_{0}^{\frac{\pi}{2}}f(\cos t)
        \dd t = \int_{0}^{\frac{\pi}{2}}f(\cos x)\dd x.
      \]
    \item 设$\matherror{x = \pi - t}$,则$\dd x = -\dd t$,且当$x = 0$时,$t =
      \pi$;当$x = \pi$时,$t = 0$,则
      \begin{align*}
        \int_{0}^{\pi}xf(\sin x)\dd x &= - \int_{\pi}^{0}(\pi - t)
        f\qty(\sin(\pi - t))\dd t\\
        &= \int_{0}^{\pi}(\pi - t)f(\sin t)\dd t\\
        &= \pi\int_{0}^{\pi}f(\sin t)\dd t - \int_{0}^{\pi}tf(\sin t)\dd t\\
        &= \pi\int_{0}^{\pi}f(\sin x)\dd x - \matherror{\int_{0}^{\pi}xf(\sin
        x)\dd x}\\
      \end{align*}
      故
      \[
        \int_{0}^{\pi}xf(\sin x)\dd x = \frac{\pi}{2}\int_{0}^{\pi}f(\sin x)
        \dd x.
      \]
  \end{solulist}
  由上述结论可得
  \begin{align*}
    \int_{0}^{\pi}\frac{x\sin x}{1 + \cos[2](x)}\dd x &= \frac{\pi}{2}\int_{0}
    ^{\pi}\frac{\sin x}{1 + \cos[2](x)}\dd x\\
    &= -\frac{\pi}{2}\int_{0}^{\pi}\frac{\dd (\cos x)}{1 + \cos[2](x)}\\
    &= \eval{-\frac{\pi}{2}\arctan(\cos x)}_0^{\pi} = \frac{\pi^2}{4}.
  \end{align*}
\end{solu}
\end{mathques}

\begin{mathques}(2){2020-10-25}{积分等式与不等式}[没有想到定积分可以当做常数看
  待,然后再凑出这个常数]
\begin{ques}
  设$f(x)$在$[-\pi, \pi]$上连续,且有$f(x) = \frac{x}{1 + \cos[2](x)} +
  \int_{-\pi}^{\pi}f(x)\sin x\dd x$,求$f(x)$.
\end{ques}
\begin{solu}
  由于$\int_{-\pi}^{\pi}f(x)\sin x\dd x$存在,记$A = \int_{-\pi}^{\pi}f(x)\sin
  x\dd x$,则$f(x) = \frac{x}{1 + \cos[2](x)} + A$,
  \[
    \int_{-\pi}^{\pi}f(x)\sin x\dd x = \int_{-\pi}^{\pi}\frac{x\sin x}{1 +
    \cos[2](x)}\dd x + \matherror{\int_{-\pi}^{\pi}A\sin x\dd x},
  \]
  令$x = \pi - t$,从而有
  \begin{align*}
    A &= \int_{-\pi}^{\pi}\frac{x\sin x}{1 + \cos[2](x)}\dd x\\
    &= 2\int_{0}^{\pi} \frac{x\sin x}{1 + \cos[2](x)}\dd x\\
    &= -2\int_{\pi}^{0}\frac{(\pi - t)\sin(\pi - t)}{1 + \cos[2](\pi - t)}\dd
    t\\
    &= 2\int_{0}^{\pi}\frac{(\pi - t)\sin t}{1 + \cos[2](t)}\dd t\\
    &= 2\pi\int_{0}^{\pi}\frac{\sin t}{1 + \cos[2](t)}\dd t - A
  \end{align*}
  故
  \[
    A = -\pi\int_{0}^{\pi}\frac{\dd (\cos t)}{1 + \cos[2](t)}
    = \eval{-\pi\arctan(\cos t)}_0^{\pi} = \frac{\pi^2}{2}
  \]
  即
  \[
    f(x) = \frac{x}{1 + \cos[2](x)} + \frac{\pi^2}{2}
  \]
\end{solu}
\end{mathques}

\begin{mathques}{2020-10-31}{多元函数微分学}[多元的极限不会求]
\begin{ques}
  函数$f(x, y) = \sqrt{\abs{xy}}$在点$(0, 0)$处(\quad).
  \begin{multichoice}
  \task 偏导数不存在
  \task \answer{偏导数存在,但不可微}
  \task 可微,但偏导数不连续
  \task 偏导数连续
  \end{multichoice}
\end{ques}
\begin{solu}
  \begin{align*}
    f'_x(0, 0) &= \lim_{\Delta x \to 0} \frac{f(0 + \Delta x, 0) - f(0, 0)}
    {\Delta x} = \lim_{\Delta x \to 0} \frac{\sqrt{\abs{\Delta x \cdot 0}} - 0}
    {\Delta x} = 0 = A,\\
    f'_y(0, 0) &= \lim_{\Delta y \to 0} \frac{f(0 + \Delta y, 0) - f(0, 0)}
    {\Delta y} = 0 = B,\\
  \end{align*}
  故$f(x, y)$在$(0, 0)$处偏导数存在,又
  \[
    \Delta z = f(0 + \Delta x, 0 + \Delta y) - f(0, 0) = \sqrt{\abs{\Delta
    x\Delta y}} - 0= \sqrt{\abs{\Delta x\Delta y}}
  \]
  故有
  \[
    \lim_{\substack{\Delta x \to 0\\\Delta y \to 0}} \frac{\Delta z
    - A\Delta x - B\Delta y}{\sqrt{(\Delta x)^2 + (\Delta y)^2}} =
    \lim_{\substack{\Delta x \to 0\\\Delta y \to 0}}
    \frac{\sqrt{\abs{\Delta x\cdot \Delta y}}}{\sqrt{(\Delta x)^2 + (\Delta y)^2}}
  \]
  不存在,故$f(x, y)$在点$(0, 0)$处不可微.
\end{solu}
\end{mathques}

\begin{mathques}{2020-10-31}{多元函数微分学}[后缀写错]
\begin{ques}
  设函数$f(x)$可微,且$f'(0) = \frac{1}{2}$,则$z = f(4x^2 - y^2)$在点$(1, 2)$
  处的全微分$\eval{\dd z}_{(1, 2)}=$\mathblank.
\end{ques}
\begin{solu}
  \answer{$4\dd x - 2\dd y$}

  令$u = 4x^2 - y^2$,则$z = f(u)$,则
  \begin{align*}
    \pdv{z}{x} &= f'(u)(4x^2 - y^2)'_x = 8xf'(u), & \eval{\pdv{z}{x}}_{(1, 2)}
    &= 8f'(0) = 4,\\
    \pdv{z}{y} &= f'(u)(4x^2 - y^2)'_y = -2yf'(u), & \eval{\pdv{z}{y}}_{(1, 2)}
    &= 8f'(0) = -2,
  \end{align*}
  故$\eval{\dd z}_{(1, 2)} = 4\dd x-2\dd y$
\end{solu}
\end{mathques}

\begin{mathques}(3){2020-11-04}{多元函数微分学}[抄错;需要合并]
\begin{ques}
  设$z = f(x^2y^2, e^{xy})$,其中$f(u, v)$有二阶连续偏导数,求$z''_{xx},
  z''_{yy}, z''_{xy}.$
\end{ques}
\begin{solu}
  令$u = x^2y^2, v = e^{xy}$,则$\pdv{u}{x} = 2xy^2, \pdv{u}{y} = 2x^2y,
  \pdv{v}{x} = ye^{xy}, \pdv{v}{y} = xe^{xy}$,故
  \begin{align*}
    \pdv{z}{x} &= \pdv{z}{u}\cdot\pdv{u}{x} + \pdv{z}{v}\cdot\pdv{v}{x}
    = \pdv{z}{u}\cdot 2xy^2 + \pdv{z}{v}\cdot ye^{xy},\\
    \pdv{z}{y} &= \pdv{z}{u}\cdot\pdv{u}{y} + \pdv{z}{v}\cdot\pdv{v}{y}
    = \pdv{z}{u}\cdot 2x^2y + \pdv{z}{v}\cdot xe^{xy},\\
  \end{align*}
  故
  \begin{align*}
    \pdv[2]{z}{x} &= \qty(\pdv[2]{z}{u}\cdot\pdv{u}{x} + \pdv{z}{u}{v}
    \cdot\pdv{v}{x})\cdot2xy^2 + \pdv{z}{u}\cdot2y^2 + \qty(\pdv{z}{u}{v}
    \cdot\pdv{u}{x} + \pdv[2]{z}{v}\cdot\pdv{v}{x})\cdot ye^{xy} + \pdv{z}{v}
    \cdot y^2e^{xy}\\
    &= \qty(\pdv[2]{z}{u}\cdot2xy^2 + \pdv{z}{u}{v}\cdot ye^{xy}) \cdot2xy^2 +
    \pdv{z}{u}\cdot2y^2 + \qty(\pdv{z}{u}{v}\cdot2xy^2 + \pdv[2]{z}{v}\cdot
    ye^{xy})\cdot ye^{xy} + \pdv{z}{v} \cdot y^2e^{xy}\\
    &= \pdv[2]{z}{u}\cdot 4x^2y^4 + \pdv[2]{z}{v}\cdot y^2e^{2xy} + \pdv{z}{u}
    {v}\cdot 4xy^3e^{xy} + \pdv{z}{u}\cdot 2y^2 + \pdv{z}{v} \cdot y^2e^{xy}
    ,\\
    \intertext{同理可得}
    \pdv[2]{z}{y} &= \pdv[2]{z}{u} \cdot 4x^4y^2 + \pdv[2]{z}{v}\cdot
    x^2e^{2xy} + \pdv{z}{u} {v}\cdot 4x^3ye^{xy} + \pdv{z}{u}\cdot 2x^2 +
    \pdv{z}{v} \cdot x^2e^{xy} ,\\
    \pdv{z}{x}{y} &= \qty(\pdv[2]{z}{u}\cdot\pdv{u}{y} + \pdv{z}{u}{v}
    \cdot\pdv{v}{y})\cdot2xy^2 + \pdv{z}{u}\cdot4xy + \qty(\pdv{z}{u}{v}
    \cdot\pdv{u}{y} + \pdv[2]{z}{v}\cdot\pdv{v}{y})\cdot ye^{xy} + \pdv{z}{v}
    \cdot (1 + xy)e^{xy}\\
    &= \qty(\pdv[2]{z}{u}\cdot2x^2y + \pdv{z}{u}{v}\cdot xe^{xy}) \cdot2xy^2 +
    \pdv{z}{u}\cdot4xy + \qty(\pdv{z}{u}{v}\cdot2x^2y + \pdv[2]{z}{v}\cdot
    xe^{xy})\cdot ye^{xy} + \pdv{z}{v} \cdot (1 + xy)e^{xy}\\
    &= \pdv[2]{z}{u}\cdot 4x^3y^3 + \pdv[2]{z}{v}\cdot xye^{2xy} + \pdv{z}{u}
    {v}\cdot 4x^2y^2e^{xy} + \pdv{z}{u}\cdot 4xy + \pdv{z}{v} \cdot (1 + xy)
    e^{xy}.
  \end{align*}
\end{solu}
\end{mathques}

\begin{mathques}{2020-10-31}{多元函数微分学}[$xy$和$xz$需要分别对$x$进行求导]
\begin{ques}
  设函数$u = f(x^2, xy, xz)$具有一阶连续偏导数,又函数$y = y(x), z = z(x)$分别
  由
  \[
    \sin xy = y\qc e^{z} = \int_{0}^{xz}\sin t\dd t
  \]
  确定,求$\dv{u}{x}$.
\end{ques}
\begin{solu}
  \begin{equation}
    \dv{u}{x} = f'_1 \cdot 2x + f'_2\cdot\matherror{\qty(y + x\dv{y}{x})} +
    f'_3\cdot\matherror{\qty(z + x\dv{z}{x})}.
    \tag{1}
  \end{equation}
  对$\sin xy = y$两边求导得
  \[
    \cos xy \cdot \qty(y + x\dv{y}{x}) = \dv{y}{x},
  \]
  故
  \begin{equation}
    \dv{y}{x} = \frac{y\cos xy}{1 - x\cos xy}.
    \tag{2}
  \end{equation}

  $e^{z} = \int_{0}^{xz} \sin t\dd t$两边对$x$求导得
  \[
    e^z\dv{z}{x} = \sin xz \cdot\qty(z + x\dv{z}{x}),
  \]
  故
  \begin{equation}
    \dv{z}{x} = \frac{z\sin xz}{e^z - x\sin xz}.
    \tag{3}
  \end{equation}
  将$(2), (3)$代入$(1)$得
  \[
    \dv{u}{x} = f'_1 \cdot 2x + f'_2 \cdot\qty(y + \frac{xy\cos xy}{1 - x\cos
    xy}) + f'_3\cdot \qty(z + \frac{xz\sin xz}{e^z - x\sin xz}).
  \]
\end{solu}
\end{mathques}

\begin{mathques}(2){2020-11-02}{多元函数微分学}[边界值直接代入$y$漏考虑了$x$的边
  界点,以后这种还是直接用拉格朗日]
\begin{ques}
  求$f(x, y) = x^2 - y^2 + 2$在椭圆域$D = \Bqty{(x, y)\bigg|x^2 + \frac{y^2}{4}
  \le 1}$上的最大值和最小值.
\end{ques}
\begin{solu}
  \begin{mathideabox}
    因为只求函数$f(x, y) = x^2 - y^2 + 2$在椭圆域$D$上的最大值和最小值,而不求
    极值,所以,只需求出$D$的内部以及$D$的边界上的驻点和导数不存在的点(不用判
    断它们是否为极值点),并求出这些点的函数值,然后比较它们的大小即可。
  \end{mathideabox}
  令$
  \begin{dcases}
  f'_x = 2x = 0,\\
  f'_y = -2y = 0,
  \end{dcases}
  $得唯一驻点$(0, 0), f(x, y)$在$D$内部没有导数不存在的点

  再求$f(x, y)$在$D$的边界上的驻点和导数不存在的点\\
  即变成了一个条件极值问题,\matherror*{边界方程$x^2 + \frac{y^2}{4} = 1$为约
  束条件},构造拉格朗日函数
  \[
    F(x, y, \lambda) = x^2 - y^2 + 2 + \lambda(x^2 + \frac{y^2}{4} - 1),
  \]
  令
  \[
  \begin{dcases}
  F'_x = 2x + 2\lambda x = 0,\\
  F'_y = -2y + \frac{\lambda}{2}y = 0,\\
  F'_{\lambda} = x^2 + \frac{y^2}{4} - 1 = 0,
  \end{dcases}
  \]
  解上述方程得四个驻点$(0, 2), (0, -2), (1, 0), (-1, 0)$,计算得
  \[
  f(0, 0) = 2\qc f(0, 2) = f(0, -2) = -2, f(1, 0) = f(-1, 0) = 3,
  \]
  故函数最大值最小值分别为$f_{\max} = 3, f_{\min} = -2.$
\end{solu}
\end{mathques}

\begin{mathques}{2020-11-03}{多元函数微分学}[不知道代换方式]
\begin{ques}
  利用变量代换$u = x, v = \frac{y}{x}$, 可将方程$x\pdv{z}{x} + y\pdv{z}{y} =
  z$化成新方程(\quad).
  \begin{multichoice}
    \task \answer{$u\pdv{z}{u} = z$}
    \task $v\pdv{z}{v} = z$
    \task $u\pdv{z}{v} = z$
    \task $v\pdv{z}{u} = z$
  \end{multichoice}
\end{ques}
\begin{solu}
  $z$是$x, y$的函数,由链式法则
  \begin{align*}
    \pdv{z}{x} &= \pdv{z}{u}\cdot \dv{u}{x} + \pdv{z}{v}\cdot \pdv{v}{x} =
    \pdv{z}{u} - \frac{y}{x^2}\cdot\pdv{z}{v}\\
    \pdv{z}{y} &= \pdv{z}{v}\cdot \pdv{v}{y} = \frac{1}{x}\cdot \pdv{z}{v}
  \end{align*}
  故$z = x\cdot \pdv{z}{u} - \frac{y}{x}\pdv{z}{v} + \frac{y}{x}\pdv{z}{v} =
  u\pdv{z}{u}$.
\end{solu}
\end{mathques}

\begin{mathques}{2020-11-03}{多元函数微分学}[不知道代换方法]
\begin{ques}
  设函数$z = xyf(\frac{y}{x})$,其中$f(u)$可导,则$x\pdv{z}{x} + y\pdv{z}{y}
  =$\mathblank.
\end{ques}
\begin{solu}
  \answer{2z}

  设$u = \frac{y}{x}$,则$z = xyf(u)$
  \begin{align*}
    \pdv{z}{x} &= yf(u) + xyf'(u)\pdv{u}{x} = yf(u) - \frac{y^2}{x}f'(u),\\
    \pdv{z}{y} &= xf(u) + xyf'(u)\pdv{u}{y} = xf(u) + yf'(u).
  \end{align*}
  故
  \[
    x\pdv{z}{x} + y\pdv{z}{y} = xyf(u) - y^2f'(u) + xyf(u) + y^2f'(u) = 2xyf(u)
    = 2z.
  \]
\end{solu}
\end{mathques}

\begin{mathques}{2020-11-03}{多元函数微分学}[二重极限不会求]
\begin{ques}
求下列极限:
\begin{queslist}
  \item $\lim_{(x, y) \to (0, 0)}\frac{\sin(x^2y)}{x^2 + y^2}.$
  \item $\lim_{(x, y) \to (0, 0)}\frac{x^2y}{x^4 + y^2}.$
\end{queslist}
\end{ques}
\begin{solu}
\begin{solulist}
  \item $\lim_{(x, y) \to (0, 0)}\frac{\sin(x^2y)}{x^2 + y^2} = \lim_{(x, y)
    \to (0, 0)} \frac{\sin(x^2y)}{x^2y}\cdot \frac{x^2}{x^2 + y^2}\cdot y$,则
    \begin{align*}
      \lim_{(x, y) \to (0, 0)} \frac{\sin(x^2y)}{x^2y} &= 1,\mathidea[r]{重要
      极限}\\
      \lim_{(x, y) \to (0, 0)} \frac{x^2}{x^2 + y^2}\cdot y &= 0,\mathidea[r]
      {无穷小乘有界函数}
    \end{align*}
    故$\lim_{(x, y) \to (0, 0)}\frac{\sin(x^2y)}{x^2 + y^2} = 1\times 0 = 0.$
  \item 令$y = kx$则
    \[
    \lim_{(x, y) \to (0, 0)}\frac{x^2y}{x^4 + y^2} = \lim_{x \to 0}
    \frac{kx^3}{x^4 + (kx)^2} = \lim_{x \to 0} \frac{kx}{x^2 + k^2} = 0.
    \]
    令$y = x^2$则
    \[
    \lim_{(x, y) \to (0, 0)}\frac{\sin(x^2y)}{x^2 + y^2} = \frac{1}{2}.
    \]
    故$\lim_{(x, y) \to (0, 0)}\frac{x^2y}{x^4 + y^2}$不存在.
\end{solulist}
\end{solu}
\end{mathques}

\begin{mathques}{2020-11-03}{多元函数微分学}[不知道代换方法,题目中的$G'_z$不
  为$0$算是提示]
\begin{ques}
  设函数$z = z(x, y)$由$G(x, y, z) = F(xy, yz) = 0$确定,其中$F$为可微函数,且
  $G'_z \neq 0$,求$x\pdv{z}{x} - y\pdv{z}{y}.$
\end{ques}
\begin{solu}
  由$G(x, y, z) = F(xy, yz)$,可得
  \begin{gather*}
  G'_x = F'_1\cdot y\qc G'_y = F'_1\cdot x + F'_2\cdot z\qc G'_z = F'_2\cdot
  y\neq 0.\\
  \end{gather*}
  $G(x, y, z) = 0$两边分别对$x, y$求偏导得
  \[
    G'_x + G'_z\pdv{z}{x} = 0\qc G'_y + G'_z\pdv{z}{y} = 0,
  \]
  故
  \begin{align*}
    \pdv{z}{x} &= -\frac{G'_x}{G'_z} = -\frac{F'_1}{F'_2}\\
    \pdv{z}{y} &= -\frac{G'_y}{G'_z} = -\frac{xF'_1 + zF'_2}{yF'_2}.
  \end{align*}
  则
  \[
    x\pdv{z}{x} - y\pdv{z}{y} = -\frac{xF'_1}{F'_2} + \frac{xF'_1 + zF'_2}
    {F'_2} = z\\
  \]
\end{solu}
\end{mathques}

\begin{mathques}{2020-11-03}{多元函数微分学}[链式法则不熟悉]
\begin{ques}
  设$x = x(y, z), y = y(z, x), z = z(x, y)$均为方程$f(x, y, z) = 0$所确定的具
  有连续偏导数的函数,证明$x'_y\cdot y'_z \cdot z'_x = -1.$
\end{ques}
\begin{solu}
  \mathproof 由$x = x(y, z)$得,$f(x(y, z), y, z) = 0$,两边对$y$求导得$f'_1
  \cdot x'_y + f'_2 = 0$,解得$x'_y = -\frac{f'_2}{f'_1}$.
  \begin{mathideabox}
    $f(x(y, z), y, z)$是看成关于$y, z$的函数,对$y$求偏导时是把$z$看成常数的因
    此不会出现$\pdv{z}{y}$,可以理解成$g'_x(x, y) = g'_x + g'_x\cdot
    \dv{y(\text{常数})}{x} $
  \end{mathideabox}
  同理$y'z = -\frac{f'_3}{f'_2}, z'_x = -\frac{f'_1}{f'_3}$,故$x'_y\cdot y'_z
  \cdot z'_x = -1.$
\end{solu}
\end{mathques}

\begin{mathques}(2){2020-11-05}{多元函数微分学}[求导符漏了]
\begin{ques}
  设$f(u)$具有二阶连续导数,且$z = xf(\frac{y}{x}) + yf(\frac{x}{y})$,求
  $\pdv{z}{x}{y}.$
\end{ques}
\begin{solu}
  设$u = \frac{y}{x}, v = \frac{x}{y}$,则
  \begin{align*}
    z &= xf(u) + yf(v),\\
    \pdv{z}{x} &= f(u) + xf'(u)\cdot \qty(-\frac{y}{x^2}) + yf'(v)\cdot
    \frac{1}{y}\\
    &= f(u) - \frac{y}{x}f'(u) + f'(v),\\
    \pdv{z}{x}{y} &= f'(u) \cdot \frac{1}{x} - \frac{1}{x}f'(u) - \frac{y}{x}
    y''(u)\cdot \frac{1}{x} + f''(v) \cdot \qty(-\frac{x}{y^2})\\
    &= -\frac{y}{x^2}f''(u) - \frac{x}{y^2}f''(v)\\
    &= -\frac{y}{x^2}f''(\frac{y}{x}) - \frac{x}{y^2}f''(\frac{x}{y}).
  \end{align*}
\end{solu}
\end{mathques}

\begin{mathques}{2020-11-03}{多元函数微分学}[求出驻点后没有判断是极大值还是极
  小值]
\begin{ques}
  求二元函数$f(x, y) = x^2y^2 + x\ln x$的极值.
\end{ques}
\begin{solu}
  \[
    f'_x(x, y) = 2xy^2 + 1 + \ln x\qc f'_y(x, y) = 2x^2y.
  \]
  令
  \[
  \begin{dcases}
    f'_x(x, y) = 0,\\
    f'_y(x, y) = 0,
  \end{dcases}
  \]
解得唯一驻点$\qty(\frac{1}{e}, 0).$由于
\begin{gather*}
  \matherror{
    f''_{xx} = 2y^2 + \frac{1}{x}\qc f''_{xy} = 4xy, f''_{yy} = 2x^2,
  }\\
  A = f''_{xx}\qty(\frac{1}{e}, 0) = e\qc B = f''_{xy}\qty(\frac{1}{e}, 0) = 0
  \qc A = f''_{yy}\qty(\frac{1}{e}, 0) = \frac{2}{e^2},
\end{gather*}
故$\matherror{B^2 - AC = - \frac{2}{e} < 0, A = e > 0}$,则$\qty(\frac{1}{e}, 0)
$为$f(x, y)$的极小值点,极小值为$-\frac{1}{e}.$
\end{solu}
\end{mathques}

\begin{mathques}{2020-11-3}{多元函数微分学}
\begin{ques}
  求函数$u = xyz$在条件$\frac{1}{x} + \frac{1}{y} + \frac{1}{z} = \frac{1}{a}
  (x > 0, y > 0, z > 0, a > 0)下的极值$.
\end{ques}
\begin{solu}
  作拉格朗日函数$L(x, y, z, \lambda) = xyz + \lambda\qty(\frac{1}{x} + \frac{1}{y}
  + \frac{1}{z} - \frac{1}{a})$.令
  \begin{empheq}[left=\empheqlbrace]{align}
    L'_x &= yz - \frac{\lambda}{x^2} = 0,\tag{1}\\
    L'_y &= xz - \frac{\lambda}{y^2} = 0,\tag{2}\\
    L'_z &= xy - \frac{\lambda}{z^2} = 0,\tag{3}
  \end{empheq}
  $(1)x + (2)y + (3)z$可得
  \begin{equation}
    3xyz - \lambda\qty(\frac{1}{x} + \frac{1}{y} + \frac{1}{z}) = 0,\tag{4}
  \end{equation}
  注意到$\frac{1}{x} + \frac{1}{y} + \frac{1}{z} = \frac{1}{a}$,$(4)$式化为
  \[
    3xyz = \frac{\lambda}{a},
  \]
  代回$(1), (2), (3)$得到$x = y = z = 3a, \lambda = 81a^4$.

  故$(3a, 3a, 3a)$是函数$u = xyz$在给定约束条件下唯一可能的极值点.

  又
  \begin{align*}
    \dd L &= xy\dd z + xz \dd y + yz \dd x - \lambda\qty(\frac{1}{x^2}\dd x +
    \frac{1}{y^2}\dd y + \frac{1}{z^2}\dd x)\\
    \dd[2]{L} &= 2(x\dd y\dd z + y\dd x\dd z + z\dd x \dd y) +
    2\lambda\qty(\frac{1}{x^3}\dd {x^2} + \frac{1}{y^3}\dd {y^2} + \frac{1}
    {z^3}\dd {z^2})\\
  \end{align*}
  又由
  \[
  \frac{1}{x} + \frac{1}{y} + \frac{1}{z} = \frac{1}{a}
  \]
  得
  \[
    \frac{1}{x^2}\dd x + \frac{1}{y^2}\dd y + \frac{1}{z^2} \dd z = 0
  \]
  即在点$(3a, 3a, 3a)$处有$\dd x + \dd y + \dd z = 0$,则
  \begin{align*}
    \eval{\dd[2]L}_{(3a, 3a, 3a)} &= 6a(\dd x\dd y + \dd y\dd z + \dd z\dd x)
    + \frac{2\lambda}{27a^3}(\dd{x^2} + \dd{y^2} + \dd{z^2})\\
    &= 6a(\dd x\dd y + \dd y\dd z + \dd z\dd x + \dd{x^2} + \dd{y^2} + \dd{z^2})
  \end{align*}

  \matherror*{要证明$\dd[2]{L} > 0$}

\end{solu}
\end{mathques}

\begin{mathques}{2020-11-06}{二重积分}[不知道怎么求$y$的范围]
\begin{ques}
  累次积分$\int_{0}^{\frac{\pi}{2}}\dd\theta\int_{0}^{\cos\theta}f(r\cos\theta,
  r\sin\theta)r\dd r$可以写成(\quad).
  \begin{multichoice}
    \task $\int_{0}^{1}\dd y\int_{0}^{\sqrt{y - y^2}}f(x, y)\dd x$
    \task $\int_{0}^{1}\dd y\int_{0}^{\sqrt{1 - y^2}}f(x, y)\dd x$
    \task $\int_{0}^{1}\dd x\int_{0}^{1}f(x, y)\dd y$
    \task \answer{$\int_{0}^{1}\dd x\int_{0}^{\sqrt{x - x^2}}f(x, y)\dd y$}
  \end{multichoice}
\end{ques}
\begin{solu}
  积分区域$D = \Bqty{(r, \theta) \bigg| 0 \le \theta \le \frac{\pi}{2}, 0 \le
  r \le \cos \theta}$

  先对$r$积分,且$0 = r_1(\theta) \le r \le r_2(\theta)$,$r_2(\theta)$为圆的
  方程$\qty(x - \frac{1}{2})^2 + y^2 = \frac{1}{4}$,然后对$\theta$积分且$0
  \le \theta \le \frac{\pi}{2}$,实际上积分区域$D$就是圆$\qty(x - \frac{1}{2})
  ^2 + y^2 = \frac{1}{4}$的上半部分。
\end{solu}
\end{mathques}

\begin{mathques}(2){2020-11-08}{二重积分}[漏了$r$导致积分算错]
\begin{ques}
求二重积分$\iint\limits_{D}\frac{1 - x^2 - y^2}{1 + x^2 + y^2}\dd x\dd y$,
其中$D$是$\{(x, y)|x^2 + y^2 \le 1\}$在第一象限的部分.
\end{ques}
\begin{solu}
  利用极坐标得
  \begin{align*}
    \text{原式} &= \int_{0}^{\frac{\pi}{2}}\dd \theta \int_{0}^{1} \frac{1 -
    r^2}{1 + r^2}\matherror{r}\dd r = \frac{\pi}{2}\int_{0}^{1}\qty(\frac{2}{1
    + r^2} - 1)r\dd r\\
    &= \eval{\frac{\pi}{2}\qty(\ln(1 + r^2) - \frac{1}{2}r^2)}_{0}^{1} =
    \frac{\pi}{2}\qty(\ln 2 - \frac{1}{2}).
  \end{align*}
\end{solu}
\end{mathques}

\begin{mathques}{2020-11-06}{二重积分}[0-1的部分算错了]
\begin{ques}
  设$F(t) = \iint\limits_{x + y \le t}f(x, y)\dd x\dd y$,其中$f(x, y) =
  \begin{dcases}
    1, & 0 \le x \le 1, 0 \le y \le 1,\\
    0, & \text{其他},\\
  \end{dcases}
  $求$F(t)$的表达式.
\end{ques}
\begin{solu}
  由被积函数$f(x, y)$的表达式可知$F(t)$在数值上等于区域$x + y \le t$与正方形$0
  \le x, y \le 1,$的公共部分的面积.

  当$t \le 0$时,$F(t) = 0$;

  当$0 < t \le 1$时,$F(t) = \frac{1}{2}t^2$;

  当$1 < t \le 2$时,$F(t) = 1 - \frac{1}{2}(2 - t)^2$;

  当$t > 2$时,$F(t) = 1.$

  综上所述,
  \[
    F(t) =
    \begin{dcases}
      0,                        & t \le 0,\\
      \frac{1}{2}t^2,           & 0 < t \le 1,\\
      -\frac{1}{2}t^2 + 2t - 1, & 1 < t \le 2,\\
      1,                        & t > 2.
    \end{dcases}
  \]
\end{solu}
\end{mathques}

\begin{mathques}(2){2020-11-18}{二重积分}[半径忘记开平方]
\begin{ques}
  设函数$f(x, y)$连续,交换二次积分次序$\int_{0}^{1}\dd y\int_{\sqrt{y}}
  ^{\sqrt{2 - y^2}}\dd x=$\mathblank.
\end{ques}
\begin{solu}
  积分区域用不等式表示为
  \[
  0 \le y \le 1\qc \sqrt{y} \le x \le \sqrt{2 - y^2}
  \]
  求解
  \[
  \begin{dcases}
  \sqrt{y} = x,\\
  x = \sqrt{2 - y^2},\\
  \end{dcases}
  \]
  得$x = 1, y = 1$

  当$0 \le x \le 1$时,$0 \le y \le x^2$;当$1 \le x \le \sqrt{2}$时,$0 \le y
  \le \sqrt{2 - x^2}$.

  故$\int_{0}^{1}\dd y\int_{\sqrt{y}} ^{\sqrt{2 - y^2}}\dd x = \int_{0}^{1}\dd
  x\int_{0}^{x^2}f(x, y)\dd y + \int_{1}^{\sqrt{2}}\dd x\int_{0}^{\sqrt{2 -
  x^2}}f(x, y)\dd y$.
\end{solu}
\end{mathques}

\begin{mathques}(3){2020-11-16}{二重积分}[范围画错]
\begin{ques}
  设$f(u)$为连续函数,则二次积分$\int_{0}^{\frac{\pi}{2}}\dd
  \theta\int_{2\cos\theta}^{2}f(r^2)r\dd r$在直角坐标系下化为二次积分
  \mathblank.
\end{ques}
\begin{solu}
  \answer{$\int_{0}^{2}\dd x\int_{\sqrt{2x - x^2}}^{\sqrt{4 - x^2}}f(x^2 + y^2)
  \dd y$}

  区域$D$在极坐标系下表示为
  \[
  0 \le \theta \le \frac{\pi}{2}\qc 2\cos\theta \le r \le 2,
  \]
  则图像在第一象限.

  \matherror*{由$r = 2\cos\theta$可得$r^2 = 2r\cos\theta$,对应$x^2 + y^2 =
  2x$}.

  \matherror*{由$r = 2$有$r^2 = 4$,对应$x^2 + y^2 = 4$}.

  故在直角坐标系下(要画图)可表示为
  \[
  0 \le x \le 2, \sqrt{2x - x^2} \le y \le \sqrt{4 - x^2}.
  \]
  即
  \[
    \text{原式} = \int_{0}^{2}\dd x\int_{\sqrt{2x - x^2}}^{\sqrt{4 - x^2}}
    f(x^2 + y^2) \dd y
  \]
\end{solu}
\end{mathques}

\begin{mathques}{2020-11-08}{二重积分}[画图画错]
\begin{ques}
  设$f(x, y)$为连续函数,直角坐标系下的二次积分$\int_{0}^{1}\dd x\int_{0}
  ^{\sqrt{x - x^2}}f(x, y)\dd y$可化为极坐标系下的二次积分\mathblank.
\end{ques}
\begin{solu}
  \answer{$\int_{0}^{\frac{\pi}{2}}\dd \theta\int_{0}^{\cos\theta}
  f(r\cos\theta, r\sin\theta)r\dd r$}

  所给积分区域$D$为
  \[
  0 \le x \le 1, 0 \le y \le \sqrt{x - x^2}.
  \]
  $y = \sqrt{x - x^2}$可化为$\qty(x - \frac{1}{2})^2 + y^2 = \qty(\frac{1}{2})
  ^2$

  画出图形,又$y = \sqrt{x - x^2}$在极坐标系下化为$r = \cos\theta$,故积分区域
  $D$可以表示为
  \[
  0 \le \theta \le \frac{\pi}{2}\qc 0 \le r \le \cos\theta
  \]
  故$\int_{0}^{1}\dd x\int_{0}^{\sqrt{x - x^2}}f(x, y)\dd y = \int_{0}
  ^{\frac{\pi}{2}}\dd \theta\int_{0}^{\cos\theta} f(r\cos\theta, r\sin\theta)
  r\dd r$
\end{solu}
\end{mathques}
\end{document}
